%% Generated by Sphinx.
\def\sphinxdocclass{report}
\documentclass[letterpaper,10pt,english]{sphinxmanual}
\ifdefined\pdfpxdimen
   \let\sphinxpxdimen\pdfpxdimen\else\newdimen\sphinxpxdimen
\fi \sphinxpxdimen=.75bp\relax

\PassOptionsToPackage{warn}{textcomp}
\usepackage[utf8]{inputenc}
\ifdefined\DeclareUnicodeCharacter
 \ifdefined\DeclareUnicodeCharacterAsOptional
  \DeclareUnicodeCharacter{"00A0}{\nobreakspace}
  \DeclareUnicodeCharacter{"2500}{\sphinxunichar{2500}}
  \DeclareUnicodeCharacter{"2502}{\sphinxunichar{2502}}
  \DeclareUnicodeCharacter{"2514}{\sphinxunichar{2514}}
  \DeclareUnicodeCharacter{"251C}{\sphinxunichar{251C}}
  \DeclareUnicodeCharacter{"2572}{\textbackslash}
 \else
  \DeclareUnicodeCharacter{00A0}{\nobreakspace}
  \DeclareUnicodeCharacter{2500}{\sphinxunichar{2500}}
  \DeclareUnicodeCharacter{2502}{\sphinxunichar{2502}}
  \DeclareUnicodeCharacter{2514}{\sphinxunichar{2514}}
  \DeclareUnicodeCharacter{251C}{\sphinxunichar{251C}}
  \DeclareUnicodeCharacter{2572}{\textbackslash}
 \fi
\fi
\usepackage{cmap}
\usepackage[T1]{fontenc}
\usepackage{amsmath,amssymb,amstext}
\usepackage{babel}
\usepackage{times}
\usepackage[Bjarne]{fncychap}
\usepackage{sphinx}

\usepackage{geometry}

% Include hyperref last.
\usepackage{hyperref}
% Fix anchor placement for figures with captions.
\usepackage{hypcap}% it must be loaded after hyperref.
% Set up styles of URL: it should be placed after hyperref.
\urlstyle{same}

\addto\captionsenglish{\renewcommand{\figurename}{Fig.}}
\addto\captionsenglish{\renewcommand{\tablename}{Table}}
\addto\captionsenglish{\renewcommand{\literalblockname}{Listing}}

\addto\captionsenglish{\renewcommand{\literalblockcontinuedname}{continued from previous page}}
\addto\captionsenglish{\renewcommand{\literalblockcontinuesname}{continues on next page}}

\addto\extrasenglish{\def\pageautorefname{page}}

\setcounter{tocdepth}{4}
\setcounter{secnumdepth}{4}


\title{Singularity Container Documentation}
\date{Dec 04, 2018}
\release{3.0}
\author{Admin Docs}
\newcommand{\sphinxlogo}{\sphinxincludegraphics{logo.png}\par}
\renewcommand{\releasename}{Release}
\makeindex

\begin{document}

\maketitle
\sphinxtableofcontents
\phantomsection\label{\detokenize{index::doc}}



\chapter{Admin Quick Start}
\label{\detokenize{admin_quickstart:admin-quick-start}}\label{\detokenize{admin_quickstart::doc}}
This document will cover installation and administration points of
\sphinxcode{\sphinxupquote{Singularity}} on a Linux host. For all other information, see the
\sphinxhref{https://www.sylabs.io/guides/3.0/user-guide/}{user guide}.

For any additional help or support contact the
\sphinxhref{https://www.sylabs.io/contact/}{Sylabs team}.


\section{Installation}
\label{\detokenize{admin_quickstart:installation}}
This section will explain the process of installing \sphinxcode{\sphinxupquote{Singularity}} from
source and building your own binary packages.


\subsection{Install Build Dependencies}
\label{\detokenize{admin_quickstart:install-build-dependencies}}
\sphinxcode{\sphinxupquote{Singularity}} requires several libraries and development tools to be
installed before you can build it from source.

\fvset{hllines={, ,}}%
\begin{sphinxVerbatim}[commandchars=\\\{\}]
\PYGZdl{} sudo yum \PYGZhy{}y update
\PYGZdl{} sudo yum \PYGZhy{}y groupinstall \PYGZdq{}Development Tools\PYGZdq{}
\PYGZdl{} sudo yum \PYGZhy{}y install git libseccomp\PYGZhy{}devel libuuid\PYGZhy{}devel openssl\PYGZhy{}devel squashfs\PYGZhy{}tools wget
\end{sphinxVerbatim}

\begin{sphinxadmonition}{note}{Note:}
Both \sphinxcode{\sphinxupquote{squashfs-tools}} and \sphinxcode{\sphinxupquote{libseccomp-devel}} are optional
dependencies but are required for full functionality.
\end{sphinxadmonition}


\subsection{Install Go}
\label{\detokenize{admin_quickstart:install-go}}
\sphinxcode{\sphinxupquote{Singularity}} is written primarily in Go, and you will need Go \textgreater{}= 1.11
installed to build it from source.

\fvset{hllines={, ,}}%
\begin{sphinxVerbatim}[commandchars=\\\{\}]
\PYGZdl{} export VERSION=1.11 OS=linux ARCH=amd64
\PYGZdl{} wget https://dl.google.com/go/go\PYGZdl{}VERSION.\PYGZdl{}OS\PYGZhy{}\PYGZdl{}ARCH.tar.gz
\PYGZdl{} sudo tar \PYGZhy{}C /usr/local \PYGZhy{}xzf go\PYGZdl{}VERSION.\PYGZdl{}OS\PYGZhy{}\PYGZdl{}ARCH.tar.gz
\end{sphinxVerbatim}

Post installation, you will need to setup your environment for Go.

\fvset{hllines={, ,}}%
\begin{sphinxVerbatim}[commandchars=\\\{\}]
\PYGZdl{} echo \PYGZsq{}export GOPATH=\PYGZdl{}\PYGZob{}HOME\PYGZcb{}/go\PYGZsq{} \PYGZgt{}\PYGZgt{} \PYGZti{}/.bashrc
\PYGZdl{} echo \PYGZsq{}export PATH=/usr/local/go/bin:\PYGZdl{}\PYGZob{}PATH\PYGZcb{}:\PYGZdl{}\PYGZob{}GOPATH\PYGZcb{}/bin\PYGZsq{} \PYGZgt{}\PYGZgt{} \PYGZti{}/.bashrc
\PYGZdl{} source \PYGZti{}/.bashrc
\end{sphinxVerbatim}

\begin{sphinxadmonition}{note}{Note:}
You may need to add the path \sphinxcode{\sphinxupquote{/usr/local/go/bin}} to the
\sphinxcode{\sphinxupquote{secure\_path}} option in your \sphinxcode{\sphinxupquote{sudoers}} config.
\end{sphinxadmonition}


\subsection{Download Source}
\label{\detokenize{admin_quickstart:download-source}}
\sphinxcode{\sphinxupquote{Singularity}} source code is available on \sphinxcode{\sphinxupquote{Github}}. You can either
download a versioned tarball from the
\sphinxhref{https://github.com/sylabs/singularity/releases}{releases page} or
clone our \sphinxcode{\sphinxupquote{git}} repository.

After you clone the \sphinxcode{\sphinxupquote{git}} repository, you can optionally \sphinxcode{\sphinxupquote{checkout}} the
\sphinxcode{\sphinxupquote{tag}} of a specific version to install (e.g. \sphinxcode{\sphinxupquote{v3.0.1}})

\fvset{hllines={, ,}}%
\begin{sphinxVerbatim}[commandchars=\\\{\}]
\PYGZdl{} mkdir \PYGZhy{}p \PYGZdl{}GOPATH/src/github.com/sylabs
\PYGZdl{} cd \PYGZdl{}GOPATH/src/github.com/sylabs
\PYGZdl{} git clone https://github.com/sylabs/singularity
\PYGZdl{} git tag \PYGZhy{}\PYGZhy{}list
\PYGZdl{} git checkout v3.0.1
\end{sphinxVerbatim}


\subsection{Configure the Build}
\label{\detokenize{admin_quickstart:configure-the-build}}
\sphinxcode{\sphinxupquote{Singularity}} uses a custom build system. You will configure the build using
the \sphinxcode{\sphinxupquote{mconfig}} script.

\begin{sphinxadmonition}{note}{Note:}
You can see all of the options for \sphinxcode{\sphinxupquote{mconfig}} by using the \sphinxcode{\sphinxupquote{-h}}
option.
\end{sphinxadmonition}

\fvset{hllines={, ,}}%
\begin{sphinxVerbatim}[commandchars=\\\{\}]
\PYGZdl{} cd \PYGZdl{}GOPATH/src/github.com/sylabs/singularity
\PYGZdl{} ./mconfig \PYGZhy{}\PYGZhy{}prefix=/usr/local \PYGZhy{}\PYGZhy{}localstatedir=/var
\end{sphinxVerbatim}


\subsection{Configuration (\sphinxstyleliteralintitle{\sphinxupquote{localstatedir}})}
\label{\detokenize{admin_quickstart:configuration-localstatedir}}
The local state directories used by \sphinxcode{\sphinxupquote{Singularity}} at runtime will be placed
under the supplied \sphinxcode{\sphinxupquote{prefix}} option. This will cause issues if that directory
tree is read-only or if it is shared between several hosts or nodes that might
run \sphinxcode{\sphinxupquote{Singularity}} simultaneously.

In such cases, you should specify the \sphinxcode{\sphinxupquote{localstatedir}} option. This will
override the \sphinxcode{\sphinxupquote{prefix}} option, instead placing the local state directories
within the path explicitly provided. Ideally this should be within the local
filesystem, specific to only a single host or node.

In the case of cluster nodes, you will need to create the following
directories on all nodes, with \sphinxcode{\sphinxupquote{root:root}} ownership and \sphinxcode{\sphinxupquote{0755}} permissions

\fvset{hllines={, ,}}%
\begin{sphinxVerbatim}[commandchars=\\\{\}]
\PYGZdl{}\PYGZob{}localstatedir\PYGZcb{}/singularity/mnt

\PYGZdl{}\PYGZob{}localstatedir\PYGZcb{}/singularity/mnt/container

\PYGZdl{}\PYGZob{}localstatedir\PYGZcb{}/singularity/mnt/final

\PYGZdl{}\PYGZob{}localstatedir\PYGZcb{}/singularity/mnt/overlay

\PYGZdl{}\PYGZob{}localstatedir\PYGZcb{}/singularity/mnt/session
\end{sphinxVerbatim}


\subsection{Build from Source}
\label{\detokenize{admin_quickstart:build-from-source}}
After you configure the build you can finish building \sphinxcode{\sphinxupquote{Singularity}} from
source.

\fvset{hllines={, ,}}%
\begin{sphinxVerbatim}[commandchars=\\\{\}]
\PYGZdl{} make \PYGZhy{}C builddir
\PYGZdl{} sudo make \PYGZhy{}C builddir install
\end{sphinxVerbatim}

\begin{sphinxadmonition}{note}{Note:}
\sphinxcode{\sphinxupquote{Singularity}} must be installed as \sphinxcode{\sphinxupquote{root}} for full functionality.
\end{sphinxadmonition}

\begin{sphinxadmonition}{note}{Note:}
\sphinxcode{\sphinxupquote{Singularity}} must be installed to a file system that allows SUID
programs for full functionality.
\end{sphinxadmonition}


\subsection{Build an RPM from Source}
\label{\detokenize{admin_quickstart:build-an-rpm-from-source}}
\begin{sphinxadmonition}{note}{Note:}
This process was greatly improved in version \sphinxcode{\sphinxupquote{3.0.1}} and we suggest
you use at least that version if you wish to build RPMs.
\end{sphinxadmonition}

You will use the \sphinxcode{\sphinxupquote{rpm}} \sphinxcode{\sphinxupquote{Makefile}} target to build a \sphinxcode{\sphinxupquote{Singularity}} RPM.

\fvset{hllines={, ,}}%
\begin{sphinxVerbatim}[commandchars=\\\{\}]
\PYGZdl{} ./mconfig
\PYGZdl{} make \PYGZhy{}C builddir rpm
\end{sphinxVerbatim}

You will find the \sphinxcode{\sphinxupquote{Singularity}} RPMs built in your home directory,
at \sphinxcode{\sphinxupquote{\textasciitilde{}/rpmbuild/}}.

If you would like to further customize the \sphinxcode{\sphinxupquote{Singularity}} installation,
you can instead use the \sphinxcode{\sphinxupquote{dist}} \sphinxcode{\sphinxupquote{Makefile}} target and run \sphinxcode{\sphinxupquote{rpmbuild}}
yourself.

\fvset{hllines={, ,}}%
\begin{sphinxVerbatim}[commandchars=\\\{\}]
\PYGZdl{} ./mconfig
\PYGZdl{} make \PYGZhy{}C builddir dist
\PYGZdl{} rpmbuild \PYGZhy{}tb \PYGZhy{}\PYGZhy{}define=\PYGZdq{}\PYGZus{}prefix /opt/singularity\PYGZdq{} singularity\PYGZhy{}*.tar.gz
\end{sphinxVerbatim}



\renewcommand{\indexname}{Index}
\printindex
\end{document}
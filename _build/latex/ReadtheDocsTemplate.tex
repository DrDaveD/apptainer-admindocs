%% Generated by Sphinx.
\def\sphinxdocclass{report}
\documentclass[letterpaper,10pt,english]{sphinxmanual}
\ifdefined\pdfpxdimen
   \let\sphinxpxdimen\pdfpxdimen\else\newdimen\sphinxpxdimen
\fi \sphinxpxdimen=.75bp\relax

\PassOptionsToPackage{warn}{textcomp}
\usepackage[utf8]{inputenc}
\ifdefined\DeclareUnicodeCharacter
 \ifdefined\DeclareUnicodeCharacterAsOptional
  \DeclareUnicodeCharacter{"00A0}{\nobreakspace}
  \DeclareUnicodeCharacter{"2500}{\sphinxunichar{2500}}
  \DeclareUnicodeCharacter{"2502}{\sphinxunichar{2502}}
  \DeclareUnicodeCharacter{"2514}{\sphinxunichar{2514}}
  \DeclareUnicodeCharacter{"251C}{\sphinxunichar{251C}}
  \DeclareUnicodeCharacter{"2572}{\textbackslash}
 \else
  \DeclareUnicodeCharacter{00A0}{\nobreakspace}
  \DeclareUnicodeCharacter{2500}{\sphinxunichar{2500}}
  \DeclareUnicodeCharacter{2502}{\sphinxunichar{2502}}
  \DeclareUnicodeCharacter{2514}{\sphinxunichar{2514}}
  \DeclareUnicodeCharacter{251C}{\sphinxunichar{251C}}
  \DeclareUnicodeCharacter{2572}{\textbackslash}
 \fi
\fi
\usepackage{cmap}
\usepackage[T1]{fontenc}
\usepackage{amsmath,amssymb,amstext}
\usepackage{babel}
\usepackage{times}
\usepackage[Bjarne]{fncychap}
\usepackage{sphinx}

\usepackage{geometry}

% Include hyperref last.
\usepackage{hyperref}
% Fix anchor placement for figures with captions.
\usepackage{hypcap}% it must be loaded after hyperref.
% Set up styles of URL: it should be placed after hyperref.
\urlstyle{same}

\addto\captionsenglish{\renewcommand{\figurename}{Fig.}}
\addto\captionsenglish{\renewcommand{\tablename}{Table}}
\addto\captionsenglish{\renewcommand{\literalblockname}{Listing}}

\addto\captionsenglish{\renewcommand{\literalblockcontinuedname}{continued from previous page}}
\addto\captionsenglish{\renewcommand{\literalblockcontinuesname}{continues on next page}}

\addto\extrasenglish{\def\pageautorefname{page}}

\setcounter{tocdepth}{4}
\setcounter{secnumdepth}{4}


\title{Singularity Container Documentation}
\date{Jun 20, 2018}
\release{1.0}
\author{Admin Docs}
\newcommand{\sphinxlogo}{\sphinxincludegraphics{logo.png}\par}
\renewcommand{\releasename}{Release}
\makeindex

\begin{document}

\maketitle
\sphinxtableofcontents
\phantomsection\label{\detokenize{index::doc}}



\chapter{Administration QuickStart}
\label{\detokenize{admin_quickstart:administration-quickstart}}\label{\detokenize{admin_quickstart::doc}}
This document will cover installation and administration points of
Singularity for multi-tenant HPC resources and will not cover usage of
the command line tools, container usage, or example use cases.


\section{Installation}
\label{\detokenize{admin_quickstart:installation}}
There are two common ways to install Singularity, from source code and
via binary packages. This document will explain the process of
installation from source, and it will depend on your build host to have
the appropriate development tools and packages installed. For Red Hat
and derivatives, you should install the following \sphinxcode{\sphinxupquote{yum}} group to ensure you
have an appropriately setup build server:

\fvset{hllines={, ,}}%
\begin{sphinxVerbatim}[commandchars=\\\{\}]
\PYGZdl{} sudo yum groupinstall \PYGZdq{}Development Tools\PYGZdq{}
\end{sphinxVerbatim}


\subsection{Downloading the source}
\label{\detokenize{admin_quickstart:downloading-the-source}}
You can download the source code either from the latest stable tarball
release or via the GitHub master repository. Here is an example
downloading and preparing the latest development code from GitHub:

\fvset{hllines={, ,}}%
\begin{sphinxVerbatim}[commandchars=\\\{\}]
\PYGZdl{} mkdir \PYGZti{}/git

\PYGZdl{} cd \PYGZti{}/git

\PYGZdl{} git clone https://github.com/singularityware/singularity.git

\PYGZdl{} cd singularity

\PYGZdl{} ./autogen.sh
\end{sphinxVerbatim}

Once you have downloaded the source, the following installation
procedures will assume you are running from the root of the source
directory.


\subsection{Source Installation}
\label{\detokenize{admin_quickstart:source-installation}}
The following example demonstrates how to install Singularity into \sphinxcode{\sphinxupquote{/usr/local}}.
You can install Singularity into any directory of your choosing, but
you must ensure that the location you select supports programs running
as \sphinxcode{\sphinxupquote{SUID}}. It is common for people to disable \sphinxcode{\sphinxupquote{SUID}} with the mount option \sphinxcode{\sphinxupquote{nosuid}} for
various network mounted file systems. To ensure proper support, it is
easiest to make sure you install Singularity to a local file system.

Assuming that \sphinxcode{\sphinxupquote{/usr/local}} is a local file system:

\fvset{hllines={, ,}}%
\begin{sphinxVerbatim}[commandchars=\\\{\}]
\PYGZdl{} ./configure \PYGZhy{}\PYGZhy{}prefix=/usr/local \PYGZhy{}\PYGZhy{}sysconfdir=/etc

\PYGZdl{} make

\PYGZdl{} sudo make install
\end{sphinxVerbatim}

\begin{sphinxadmonition}{note}{Note:}
\sphinxstylestrong{The} \sphinxcode{\sphinxupquote{make install}} \sphinxstylestrong{above must be run as root to have Singularity properly
installed. Failure to install as root will cause Singularity to not
function properly or have limited functionality when run by a non-root
user.}
\end{sphinxadmonition}

Also note that when you configure, \sphinxcode{\sphinxupquote{squashfs-tools}} is \sphinxstylestrong{not} required, however it is
required for full functionality. You will see this message after the
configuration:

\fvset{hllines={, ,}}%
\begin{sphinxVerbatim}[commandchars=\\\{\}]
mksquashfs from squash\PYGZhy{}tools is required for full functionality
\end{sphinxVerbatim}

If you choose not to install \sphinxcode{\sphinxupquote{squashfs-tools}}, you will hit an error when your users try
a pull from Docker Hub, for example.


\subsection{Prefix in special places (\textendash{}localstatedir)}
\label{\detokenize{admin_quickstart:prefix-in-special-places-localstatedir}}
As with most autotools-based build scripts, you are able to supply the \sphinxcode{\sphinxupquote{-{-}prefix}}
argument to the configure script to change where Singularity will be
installed. Care must be taken when this path is not a local filesystem
or has atypical permissions. The local state directories used by
Singularity at runtime will also be placed under the supplied \sphinxcode{\sphinxupquote{-{-}prefix}} and this
will cause malfunction if the tree is read-only. You may also
experience issues if this directory is shared between several
hosts/nodes that might run Singularity simultaneously.

In such cases, you should specify the \sphinxcode{\sphinxupquote{-{-}localstatedir}} variable in addition to \sphinxcode{\sphinxupquote{-{-}prefix}}. This
will override the prefix, instead placing the local state directories
within the path explicitly provided. Ideally this should be within the
local filesystem, specific to only a single host or node.
For example, the Makefile contains this variable by default:

\fvset{hllines={, ,}}%
\begin{sphinxVerbatim}[commandchars=\\\{\}]
CONTAINER\PYGZus{}OVERLAY = \PYGZdl{}\PYGZob{}prefix\PYGZcb{}/var/singularity/mnt/overlay
\end{sphinxVerbatim}

By supplying the configure argument \sphinxcode{\sphinxupquote{-{-}localstatedir=/some/other/place}} Singularity will instead be built
with the following. Note that \sphinxcode{\sphinxupquote{\$\{prefix\}/var}} that has been replaced by the supplied
value:

\fvset{hllines={, ,}}%
\begin{sphinxVerbatim}[commandchars=\\\{\}]
CONTAINER\PYGZus{}OVERLAY = /some/other/place/singularity/mnt/overlay
\end{sphinxVerbatim}

In the case of cluster nodes, you will need to ensure the following
directories are created on all nodes, with \sphinxcode{\sphinxupquote{root:root}} ownership and \sphinxcode{\sphinxupquote{0755}} permissions:

\fvset{hllines={, ,}}%
\begin{sphinxVerbatim}[commandchars=\\\{\}]
\PYGZdl{}\PYGZob{}localstatedir\PYGZcb{}/singularity/mnt

\PYGZdl{}\PYGZob{}localstatedir\PYGZcb{}/singularity/mnt/container

\PYGZdl{}\PYGZob{}localstatedir\PYGZcb{}/singularity/mnt/final

\PYGZdl{}\PYGZob{}localstatedir\PYGZcb{}/singularity/mnt/overlay

\PYGZdl{}\PYGZob{}localstatedir\PYGZcb{}/singularity/mnt/session
\end{sphinxVerbatim}

Singularity will fail to execute without these directories. They are
normally created by the install make target; when using a local
directory for \sphinxcode{\sphinxupquote{-{-}localstatedir}} these will only be created on the node \sphinxcode{\sphinxupquote{make}} is run on.


\subsection{Building an RPM directly from the source}
\label{\detokenize{admin_quickstart:building-an-rpm-directly-from-the-source}}
Singularity includes all of the necessary bits to properly create an RPM
package directly from the source tree, and you can create an RPM by
doing the following:

\fvset{hllines={, ,}}%
\begin{sphinxVerbatim}[commandchars=\\\{\}]
\PYGZdl{} ./configure

\PYGZdl{} make dist

\PYGZdl{} rpmbuild \PYGZhy{}ta singularity\PYGZhy{}*.tar.gz
\end{sphinxVerbatim}

Near the bottom of the build output you will see several lines like:

\fvset{hllines={, ,}}%
\begin{sphinxVerbatim}[commandchars=\\\{\}]
...
Wrote: /home/gmk/rpmbuild/SRPMS/singularity\PYGZhy{}2.3.el7.centos.src.rpm

Wrote: /home/gmk/rpmbuild/RPMS/x86\PYGZus{}64/singularity\PYGZhy{}2.3.el7.centos.x86\PYGZus{}64.rpm

Wrote: /home/gmk/rpmbuild/RPMS/x86\PYGZus{}64/singularity\PYGZhy{}devel\PYGZhy{}2.3.el7.centos.x86\PYGZus{}64.rpm

Wrote: /home/gmk/rpmbuild/RPMS/x86\PYGZus{}64/singularity\PYGZhy{}debuginfo\PYGZhy{}2.3.el7.centos.x86\PYGZus{}64.rpm

...
\end{sphinxVerbatim}

You will want to identify the appropriate path to the binary RPM that
you wish to install, in the above example the package we want to install
is \sphinxcode{\sphinxupquote{singularity-2.3.el7.centos.x86\_64.rpm}} , and you should install it with the following command:

\fvset{hllines={, ,}}%
\begin{sphinxVerbatim}[commandchars=\\\{\}]
\PYGZdl{} sudo yum install /home/gmk/rpmbuild/RPMS/x86\PYGZus{}64/singularity\PYGZhy{}2.3.el7.centos.x86\PYGZus{}64.rpm
\end{sphinxVerbatim}

\begin{sphinxadmonition}{note}{Note:}
If you want to have the binary RPM install the files to an
alternative location, you should define the environment variable
‘PREFIX’ (below) to suit your needs, and use the following command to
build:
\end{sphinxadmonition}

\fvset{hllines={, ,}}%
\begin{sphinxVerbatim}[commandchars=\\\{\}]
\PYGZdl{} PREFIX=/opt/singularity

\PYGZdl{} rpmbuild \PYGZhy{}ta \PYGZhy{}\PYGZhy{}define=\PYGZdq{}\PYGZus{}prefix \PYGZdl{}PREFIX\PYGZdq{} \PYGZhy{}\PYGZhy{}define \PYGZdq{}\PYGZus{}sysconfdir \PYGZdl{}PREFIX/etc\PYGZdq{} \PYGZhy{}\PYGZhy{}define \PYGZdq{}\PYGZus{}defaultdocdir \PYGZdl{}PREFIX/share\PYGZdq{} singularity\PYGZhy{}*.tar.gz
\end{sphinxVerbatim}

We recommend you look at our {\hyperref[\detokenize{security:security}]{\sphinxcrossref{\DUrole{std,std-ref}{security admin guide}}}} to get further information about container
privileges and mounting.


\chapter{Security}
\label{\detokenize{security:security}}\label{\detokenize{security:id1}}\label{\detokenize{security::doc}}

\section{Container security paradigms}
\label{\detokenize{security:container-security-paradigms}}
First some background. Most container platforms operate on the
premise, \sphinxstylestrong{trusted users running trusted containers}. This means that
the primary UNIX account controlling the container platform is either
“root” or user(s) that root has deputized (either via \sphinxcode{\sphinxupquote{sudo}} or given access
to a control socket of a root owned daemon process).
Singularity on the other hand, operates on a different premise because
it was developed for HPC type infrastructures where you have users,
none of which are considered trusted. This means the paradigm is
considerably different as we must support \sphinxstylestrong{untrusted users running
untrusted containers}.


\section{Untrusted users running untrusted containers!}
\label{\detokenize{security:untrusted-users-running-untrusted-containers}}
This simple phrase describes the security perspective Singularity is
designed with. And if you additionally consider the fact that running
containers at all typically requires some level of privilege
escalation, means that attention to security is of the utmost
importance.


\subsection{Privilege escalation is necessary for containerization!}
\label{\detokenize{security:privilege-escalation-is-necessary-for-containerization}}
As mentioned, there are several containerization system calls and
functions which are considered “privileged” in that they must be
executed with a certain level of capability/privilege. To do this, all
container systems must employ one of the following mechanisms:
\begin{enumerate}
\item {} 
\sphinxstylestrong{Limit usage to root:} Only allow the root user (or users granted \sphinxcode{\sphinxupquote{sudo}})
to run containers. This has the obvious limitation of not allowing
arbitrary users the ability to run containers, nor does it allow
users to run containers as themselves. Access to data, security data,
and securing systems becomes difficult and perhaps impossible.

\item {} 
\sphinxstylestrong{Root owned daemon process:} Some container systems use a root
owned daemon background process which manages the containers and
spawns the jobs within the container. Implementations of this
typically have an IPC control socket for communicating with this root
owned daemon process and if you wish to allow trusted users to
control the daemon, you must give them access to the control socket.
This is the Docker model.

\item {} 
\sphinxstylestrong{SetUID:} Set UID is the “old school” UNIX method for running a
particular program with escalated permission. While it is widely used
due to it’s legacy and POSIX requirement, it lacks the ability to
manage fine grained control of what a process can and can not do; a
SetUID root program runs as root with all capabilities that comes
with root. For this reason, SetUID programs are traditional targets
for hackers.

\item {} 
\sphinxstylestrong{User Namespace:} The Linux kernel’s user namespace may allow a
user to virtually become another user and run a limited set
privileged system functions. Here the privilege escalation is managed
via the Linux kernel which takes the onus off of the program. This is
a new kernel feature and thus requires new kernels and not all
distributions have equally adopted this technology.

\item {} 
\sphinxstylestrong{Capability Sets:} Linux handles permissions, access, and roles via
capability sets. The root user has these capabilities automatically
activated while non-privileged users typically do not have these
capabilities enabled. You can enable and disable capabilities on a
per process and per file basis (if allowed to do so).

\end{enumerate}


\subsection{How does Singularity do it?}
\label{\detokenize{security:how-does-singularity-do-it}}
Singularity must allow users to run containers as themselves which rules
out options 1 and 2 from the above list. Singularity supports the rest
of the options to following degrees of functionally:
\begin{itemize}
\item {} 
\sphinxstylestrong{User Namespace:} Singularity supports the user namespace natively
and can run completely unprivileged (“rootless”) since version 2.2
(October 2016) but features are severely limited. You will not be
able to use container “images” and will be forced to only work with
directory (sandbox) based containers. Additionally, as mentioned, the
user namespace is not equally supported on all distribution kernels
so don’t count on legacy system support and usability may vary.

\item {} 
\sphinxstylestrong{SetUID:} This is the default usage model for Singularity because
it gives the most flexibility in terms of supported features and
legacy compliance. It is also the most risky from a security
perspective. For that reason, Singularity has been developed with
transparency in mind. The code is written with attention to
simplicity and readability and Singularity increases the effective
permission set only when it is necessary, and drops it immediately
(as can be seen with the \sphinxcode{\sphinxupquote{\textendash{}debug}} run flag). There have been several
independent audits of the source code, and while they are not
definitive, it is a good assurance.

\item {} 
\sphinxstylestrong{Capability Sets:} This is where Singularity is headed as an
alternative to SetUID because it allows for much finer grained
capability control and will support all of Singularity’s features.
The downside is that it is not supported equally on shared file
systems.

\end{itemize}


\section{Where are the Singularity priviledged components}
\label{\detokenize{security:where-are-the-singularity-priviledged-components}}
When you install Singularity as root, it will automatically setup the
necessary files as SetUID (as of version 2.4, this is the default run
mode). The location of these files is dependent on how Singularity was
installed and the options passed to the \sphinxcode{\sphinxupquote{configure}} script. Assuming a default \sphinxcode{\sphinxupquote{./configure}} run
which installs files into \sphinxcode{\sphinxupquote{-{-}prefix}} of \sphinxcode{\sphinxupquote{/usr/local}} you can find the SetUID programs as
follows:

\fvset{hllines={, ,}}%
\begin{sphinxVerbatim}[commandchars=\\\{\}]
\PYGZdl{} find /usr/local/libexec/singularity/ \PYGZhy{}perm \PYGZhy{}4000

/usr/local/libexec/singularity/bin/start\PYGZhy{}suid

/usr/local/libexec/singularity/bin/action\PYGZhy{}suid

/usr/local/libexec/singularity/bin/mount\PYGZhy{}suid
\end{sphinxVerbatim}

Each of the binaries is named accordingly to the action that it is
suited for, and generally, each handles the required privilege
escalation necessary for Singularity to operate. What specifically
requires escalated privileges?
\begin{enumerate}
\item {} 
Mounting (and looping) the Singularity container image

\item {} 
Creation of the necessary namespaces in the kernel

\item {} 
Binding host paths into the container

\end{enumerate}

Removing any of these SUID binaries or changing the permissions on them
would cause Singularity to utilize the non-SUID workflows. Each file
with \sphinxcode{\sphinxupquote{*-suid}} also has a non-suid equivalent:

\fvset{hllines={, ,}}%
\begin{sphinxVerbatim}[commandchars=\\\{\}]
/usr/local/libexec/singularity/bin/start

/usr/local/libexec/singularity/bin/action

/usr/local/libexec/singularity/bin/mount
\end{sphinxVerbatim}

While most of these workflows will not properly function without the
SUID components, we have provided these fall back executables for
sites that wish to limit the SETUID capabilities to the bare
essentials/minimum. To disable the SetUID portions of Singularity, you
can either remove the above \sphinxcode{\sphinxupquote{*-suid}} files, or you can edit the setting for \sphinxcode{\sphinxupquote{allow suid}} at
the top of the \sphinxcode{\sphinxupquote{singularity.conf}} file, which is typically located in \sphinxcode{\sphinxupquote{\$PREFIX/etc/singularity/singularity.conf}}.

\fvset{hllines={, ,}}%
\begin{sphinxVerbatim}[commandchars=\\\{\}]
\PYGZsh{} ALLOW SETUID: [BOOL]

\PYGZsh{} DEFAULT: yes

\PYGZsh{} Should we allow users to utilize the setuid program flow within Singularity?

\PYGZsh{} note1: This is the default mode, and to utilize all features, this option

\PYGZsh{} will need to be enabled.

\PYGZsh{} note2: If this option is disabled, it will rely on the user namespace

\PYGZsh{} exclusively which has not been integrated equally between the different

\PYGZsh{} Linux distributions.

allow setuid = yes
\end{sphinxVerbatim}

You can also install Singularity as root without any of the SetUID
components with the configure option \sphinxcode{\sphinxupquote{-{-}disable-suid}} as follows:

\fvset{hllines={, ,}}%
\begin{sphinxVerbatim}[commandchars=\\\{\}]
\PYGZdl{} ./configure \PYGZhy{}\PYGZhy{}disable\PYGZhy{}suid \PYGZhy{}\PYGZhy{}prefix=/usr/local

\PYGZdl{} make

\PYGZdl{} sudo make install
\end{sphinxVerbatim}


\section{Can I install Singularity as a user?}
\label{\detokenize{security:can-i-install-singularity-as-a-user}}
Yes, but don’t expect all of the functions to work. If the SetUID
components are not present, Singularity will attempt to use the “user
namespace”. Even if the kernel you are using supports this namespace
fully, you will still not be able to access all of the Singularity
features.


\section{Container permissions and usage strategy}
\label{\detokenize{security:container-permissions-and-usage-strategy}}
As a system admin, you want to set up a configuration that is
customized for your cluster or shared resource. In the following
paragraphs, we will elaborate on this container permissions strategy,
giving detail about which users are allowed to run containers, along
with image curation and ownership.

These settings can all be found in the Singularity configuration file
which is installed to \sphinxcode{\sphinxupquote{\$PREFIX/etc/singularity/singularity.conf}}. When running in a privileged mode, the
configuration file \sphinxstylestrong{MUST} be owned by root and thus the system
administrator always has the final control.


\subsection{controlling what kind of containers are allowed}
\label{\detokenize{security:controlling-what-kind-of-containers-are-allowed}}
Singularity supports several different container formats:
\begin{itemize}
\item {} 
\sphinxstylestrong{squashfs:} Compressed immutable (read only) container images
(default in version 2.4)

\item {} 
\sphinxstylestrong{extfs:} Raw file system writable container images

\item {} 
\sphinxstylestrong{dir:} Sandbox containers (chroot style directories)

\end{itemize}

Using the Singularity configuration file, you can control what types of
containers Singularity will support:

\fvset{hllines={, ,}}%
\begin{sphinxVerbatim}[commandchars=\\\{\}]
\PYGZsh{} ALLOW CONTAINER \PYGZdl{}\PYGZob{}TYPE\PYGZcb{}: [BOOL]

\PYGZsh{} DEFAULT: yes

\PYGZsh{} This feature limits what kind of containers that Singularity will allow

\PYGZsh{} users to use (note this does not apply for root).

allow container squashfs = yes

allow container extfs = yes

allow container dir = yes
\end{sphinxVerbatim}


\subsection{limiting usage to specific container file owners}
\label{\detokenize{security:limiting-usage-to-specific-container-file-owners}}
One benefit of using container images is that they exist on the
filesystem as any other file would. This means that POSIX permissions
are mandatory. Here you can configure Singularity to only “trust”
containers that are owned by a particular set of users.

\fvset{hllines={, ,}}%
\begin{sphinxVerbatim}[commandchars=\\\{\}]
\PYGZsh{} LIMIT CONTAINER OWNERS: [STRING]

\PYGZsh{} DEFAULT: NULL

\PYGZsh{} Only allow containers to be used that are owned by a given user. If this

\PYGZsh{} configuration is undefined (commented or set to NULL), all containers are

\PYGZsh{} allowed to be used. This feature only applies when Singularity is running in

\PYGZsh{} SUID mode and the user is non\PYGZhy{}root.

\PYGZsh{}limit container owners = gmk, singularity, nobody
\end{sphinxVerbatim}

\begin{sphinxadmonition}{note}{Note:}
If you are in a high risk security environment, you may want to
enable this feature. Trusting container images to users could allow a
malicious user to modify an image either before or while being used and
cause unexpected behavior from the kernel (e.g. a \sphinxhref{https://en.wikipedia.org/wiki/Denial-of-service\_attack}{DOS
attack}). For
more information, please see: \sphinxurl{https://lwn.net/Articles/652468/}
\end{sphinxadmonition}


\subsection{limiting usage to specific paths}
\label{\detokenize{security:limiting-usage-to-specific-paths}}
The configuration file also gives you the ability to limit containers to
specific paths. This is very useful to ensure that only trusted or
blessed container’s are being used (it is also beneficial to ensure that
containers are only being used on performant file systems).

\fvset{hllines={, ,}}%
\begin{sphinxVerbatim}[commandchars=\\\{\}]
\PYGZsh{} LIMIT CONTAINER PATHS: [STRING]

\PYGZsh{} DEFAULT: NULL

\PYGZsh{} Only allow containers to be used that are located within an allowed path

\PYGZsh{} prefix. If this configuration is undefined (commented or set to NULL),

\PYGZsh{} containers will be allowed to run from anywhere on the file system. This

\PYGZsh{} feature only applies when Singularity is running in SUID mode and the user is

\PYGZsh{} non\PYGZhy{}root.

\PYGZsh{}limit container paths = /scratch, /tmp, /global
\end{sphinxVerbatim}


\section{Logging}
\label{\detokenize{security:logging}}
Singularity offers a very comprehensive auditing mechanism via the
system log. For each command that is issued, it prints the UID, PID, and
location of the command. For example, let’s see what happens if we shell
into an image:

\fvset{hllines={, ,}}%
\begin{sphinxVerbatim}[commandchars=\\\{\}]
\PYGZdl{} singularity exec ubuntu true

\PYGZdl{} singularity shell \PYGZhy{}\PYGZhy{}home \PYGZdl{}HOME:/ ubuntu

Singularity: Invoking an interactive shell within container...


ERROR  : Failed to execv() /.singularity.d/actions/shell, continuing to /bin/sh: No such file or directory

ERROR  : What are you doing gmk, this is highly irregular!

ABORT  : Retval = 255
\end{sphinxVerbatim}

We can then peek into the system log to see what was recorded:

\fvset{hllines={, ,}}%
\begin{sphinxVerbatim}[commandchars=\\\{\}]
Oct  5 08:51:12 localhost Singularity: action\PYGZhy{}suid (U=1000,P=32320)\PYGZgt{} USER=gmk, IMAGE=\PYGZsq{}ubuntu\PYGZsq{}, COMMAND=\PYGZsq{}exec\PYGZsq{}

Oct  5 08:53:13 localhost Singularity: action\PYGZhy{}suid (U=1000,P=32311)\PYGZgt{} USER=gmk, IMAGE=\PYGZsq{}ubuntu\PYGZsq{}, COMMAND=\PYGZsq{}shell\PYGZsq{}

Oct  5 08:53:13 localhost Singularity: action\PYGZhy{}suid (U=1000,P=32311)\PYGZgt{} Failed to execv() /.singularity.d/actions/shell, continuing to /bin/sh: No such file or directory

Oct  5 08:53:13 localhost Singularity: action\PYGZhy{}suid (U=1000,P=32311)\PYGZgt{} What are you doing gmk, this is highly irregular!

Oct  5 08:53:13 localhost Singularity: action\PYGZhy{}suid (U=1000,P=32311)\PYGZgt{} Retval = 255
\end{sphinxVerbatim}


\subsection{A peek into the SetUID program flow}
\label{\detokenize{security:a-peek-into-the-setuid-program-flow}}
We can also add the \sphinxcode{\sphinxupquote{-{-}debug}} argument to any command itself at runtime to see
everything that Singularity is doing. In this case we can run
Singularity in debug mode and request use of the PID namespace so we can
see what Singularity is doing there:

\fvset{hllines={, ,}}%
\begin{sphinxVerbatim}[commandchars=\\\{\}]
\PYGZdl{} singularity \PYGZhy{}\PYGZhy{}debug shell \PYGZhy{}\PYGZhy{}pid ubuntu

Enabling debugging

Ending argument loop

Singularity version: 2.3.9\PYGZhy{}development.gc35b753

Exec\PYGZsq{}ing: /usr/local/libexec/singularity/cli/shell.exec

Evaluating args: \PYGZsq{}\PYGZhy{}\PYGZhy{}pid ubuntu\PYGZsq{}
\end{sphinxVerbatim}

(snipped to PID namespace implementation)

\fvset{hllines={, ,}}%
\begin{sphinxVerbatim}[commandchars=\\\{\}]
DEBUG   [U=1000,P=30961]   singularity\PYGZus{}runtime\PYGZus{}ns\PYGZus{}pid()              Using PID namespace: CLONE\PYGZus{}NEWPID

DEBUG   [U=1000,P=30961]   singularity\PYGZus{}runtime\PYGZus{}ns\PYGZus{}pid()              Virtualizing PID namespace

DEBUG   [U=1000,P=30961]   singularity\PYGZus{}registry\PYGZus{}get()                Returning NULL on \PYGZsq{}DAEMON\PYGZus{}START\PYGZsq{}

DEBUG   [U=1000,P=30961]   prepare\PYGZus{}fork()                            Creating parent/child coordination pipes.

VERBOSE [U=1000,P=30961]   singularity\PYGZus{}fork()                        Forking child process

DEBUG   [U=1000,P=30961]   singularity\PYGZus{}priv\PYGZus{}escalate()               Temporarily escalating privileges (U=1000)

DEBUG   [U=0,P=30961]      singularity\PYGZus{}priv\PYGZus{}escalate()               Clearing supplementary GIDs.

DEBUG   [U=0,P=30961]      singularity\PYGZus{}priv\PYGZus{}drop()                   Dropping privileges to UID=1000, GID=1000 (8 supplementary GIDs)

DEBUG   [U=0,P=30961]      singularity\PYGZus{}priv\PYGZus{}drop()                   Restoring supplementary groups

DEBUG   [U=1000,P=30961]   singularity\PYGZus{}priv\PYGZus{}drop()                   Confirming we have correct UID/GID

VERBOSE [U=1000,P=30961]   singularity\PYGZus{}fork()                        Hello from parent process

DEBUG   [U=1000,P=30961]   install\PYGZus{}generic\PYGZus{}signal\PYGZus{}handle()           Assigning generic sigaction()s

DEBUG   [U=1000,P=30961]   install\PYGZus{}generic\PYGZus{}signal\PYGZus{}handle()           Creating generic signal pipes

DEBUG   [U=1000,P=30961]   install\PYGZus{}sigchld\PYGZus{}signal\PYGZus{}handle()           Assigning SIGCHLD sigaction()

DEBUG   [U=1000,P=30961]   install\PYGZus{}sigchld\PYGZus{}signal\PYGZus{}handle()           Creating sigchld signal pipes

DEBUG   [U=1000,P=30961]   singularity\PYGZus{}fork()                        Dropping permissions

DEBUG   [U=0,P=30961]      singularity\PYGZus{}priv\PYGZus{}drop()                   Dropping privileges to UID=1000, GID=1000 (8 supplementary GIDs)

DEBUG   [U=0,P=30961]      singularity\PYGZus{}priv\PYGZus{}drop()                   Restoring supplementary groups

DEBUG   [U=1000,P=30961]   singularity\PYGZus{}priv\PYGZus{}drop()                   Confirming we have correct UID/GID

DEBUG   [U=1000,P=30961]   singularity\PYGZus{}signal\PYGZus{}go\PYGZus{}ahead()             Sending go\PYGZhy{}ahead signal: 0

DEBUG   [U=1000,P=30961]   wait\PYGZus{}child()                              Parent process is waiting on child process

DEBUG   [U=0,P=1]          singularity\PYGZus{}priv\PYGZus{}drop()                   Dropping privileges to UID=1000, GID=1000 (8 supplementary GIDs)

DEBUG   [U=0,P=1]          singularity\PYGZus{}priv\PYGZus{}drop()                   Restoring supplementary groups

DEBUG   [U=1000,P=1]       singularity\PYGZus{}priv\PYGZus{}drop()                   Confirming we have correct UID/GID

VERBOSE [U=1000,P=1]       singularity\PYGZus{}fork()                        Hello from child process

DEBUG   [U=1000,P=1]       singularity\PYGZus{}wait\PYGZus{}for\PYGZus{}go\PYGZus{}ahead()           Waiting for go\PYGZhy{}ahead signal

DEBUG   [U=1000,P=1]       singularity\PYGZus{}wait\PYGZus{}for\PYGZus{}go\PYGZus{}ahead()           Received go\PYGZhy{}ahead signal: 0

VERBOSE [U=1000,P=1]       singularity\PYGZus{}registry\PYGZus{}set()                Adding value to registry: \PYGZsq{}PIDNS\PYGZus{}ENABLED\PYGZsq{} = \PYGZsq{}1\PYGZsq{}
\end{sphinxVerbatim}

(snipped to end)

\fvset{hllines={, ,}}%
\begin{sphinxVerbatim}[commandchars=\\\{\}]
DEBUG   [U=1000,P=1]       envar\PYGZus{}set()                               Unsetting environment variable: SINGULARITY\PYGZus{}APPNAME

DEBUG   [U=1000,P=1]       singularity\PYGZus{}registry\PYGZus{}get()                Returning value from registry: \PYGZsq{}COMMAND\PYGZsq{} = \PYGZsq{}shell\PYGZsq{}

LOG     [U=1000,P=1]       main()                                    USER=gmk, IMAGE=\PYGZsq{}ubuntu\PYGZsq{}, COMMAND=\PYGZsq{}shell\PYGZsq{}

INFO    [U=1000,P=1]       action\PYGZus{}shell()                            Singularity: Invoking an interactive shell within container...


DEBUG   [U=1000,P=1]       action\PYGZus{}shell()                            Exec\PYGZsq{}ing /.singularity.d/actions/shell

Singularity ubuntu:\PYGZti{}\PYGZgt{}
\end{sphinxVerbatim}

Not only do I see all of the configuration options that I (probably
forgot about) previously set, I can trace the entire flow of Singularity
from the first execution of an action (shell) to the final shell into
the container. Each line also describes what is the effective UID
running the command, what is the PID, and what is the function emitting
the debug message.


\subsection{A peek into the “rootless” program flow}
\label{\detokenize{security:a-peek-into-the-rootless-program-flow}}
The above snippet was using the default SetUID program flow with a
container image file named “ubuntu”. For comparison, if we also use the \sphinxcode{\sphinxupquote{-{-}userns}}
flag, and snip in the same places, you can see how the effective UID is
never escalated, but we have the same outcome using a sandbox directory
(chroot) style container.

\fvset{hllines={, ,}}%
\begin{sphinxVerbatim}[commandchars=\\\{\}]
\PYGZdl{} singularity \PYGZhy{}d shell \PYGZhy{}\PYGZhy{}pid \PYGZhy{}\PYGZhy{}userns ubuntu.dir/

Enabling debugging

Ending argument loop

Singularity version: 2.3.9\PYGZhy{}development.gc35b753

Exec\PYGZsq{}ing: /usr/local/libexec/singularity/cli/shell.exec

Evaluating args: \PYGZsq{}\PYGZhy{}\PYGZhy{}pid \PYGZhy{}\PYGZhy{}userns ubuntu.dir/\PYGZsq{}
\end{sphinxVerbatim}

(snipped to PID namespace implementation, same place as above)

\fvset{hllines={, ,}}%
\begin{sphinxVerbatim}[commandchars=\\\{\}]
DEBUG   [U=1000,P=32081]   singularity\PYGZus{}runtime\PYGZus{}ns\PYGZus{}pid()              Using PID namespace: CLONE\PYGZus{}NEWPID

DEBUG   [U=1000,P=32081]   singularity\PYGZus{}runtime\PYGZus{}ns\PYGZus{}pid()              Virtualizing PID namespace

DEBUG   [U=1000,P=32081]   singularity\PYGZus{}registry\PYGZus{}get()                Returning NULL on \PYGZsq{}DAEMON\PYGZus{}START\PYGZsq{}

DEBUG   [U=1000,P=32081]   prepare\PYGZus{}fork()                            Creating parent/child coordination pipes.

VERBOSE [U=1000,P=32081]   singularity\PYGZus{}fork()                        Forking child process

DEBUG   [U=1000,P=32081]   singularity\PYGZus{}priv\PYGZus{}escalate()               Not escalating privileges, user namespace enabled

DEBUG   [U=1000,P=32081]   singularity\PYGZus{}priv\PYGZus{}drop()                   Not dropping privileges, user namespace enabled

VERBOSE [U=1000,P=32081]   singularity\PYGZus{}fork()                        Hello from parent process

DEBUG   [U=1000,P=32081]   install\PYGZus{}generic\PYGZus{}signal\PYGZus{}handle()           Assigning generic sigaction()s

DEBUG   [U=1000,P=32081]   install\PYGZus{}generic\PYGZus{}signal\PYGZus{}handle()           Creating generic signal pipes

DEBUG   [U=1000,P=32081]   install\PYGZus{}sigchld\PYGZus{}signal\PYGZus{}handle()           Assigning SIGCHLD sigaction()

DEBUG   [U=1000,P=32081]   install\PYGZus{}sigchld\PYGZus{}signal\PYGZus{}handle()           Creating sigchld signal pipes

DEBUG   [U=1000,P=32081]   singularity\PYGZus{}signal\PYGZus{}go\PYGZus{}ahead()             Sending go\PYGZhy{}ahead signal: 0

DEBUG   [U=1000,P=32081]   wait\PYGZus{}child()                              Parent process is waiting on child process

DEBUG   [U=1000,P=1]       singularity\PYGZus{}priv\PYGZus{}drop()                   Not dropping privileges, user namespace enabled

VERBOSE [U=1000,P=1]       singularity\PYGZus{}fork()                        Hello from child process

DEBUG   [U=1000,P=1]       singularity\PYGZus{}wait\PYGZus{}for\PYGZus{}go\PYGZus{}ahead()           Waiting for go\PYGZhy{}ahead signal

DEBUG   [U=1000,P=1]       singularity\PYGZus{}wait\PYGZus{}for\PYGZus{}go\PYGZus{}ahead()           Received go\PYGZhy{}ahead signal: 0

VERBOSE [U=1000,P=1]       singularity\PYGZus{}registry\PYGZus{}set()                Adding value to registry: \PYGZsq{}PIDNS\PYGZus{}ENABLED\PYGZsq{} = \PYGZsq{}1\PYGZsq{}
\end{sphinxVerbatim}

(snipped to end)

\fvset{hllines={, ,}}%
\begin{sphinxVerbatim}[commandchars=\\\{\}]
DEBUG   [U=1000,P=1]       envar\PYGZus{}set()                               Unsetting environment variable: SINGULARITY\PYGZus{}APPNAME

DEBUG   [U=1000,P=1]       singularity\PYGZus{}registry\PYGZus{}get()                Returning value from registry: \PYGZsq{}COMMAND\PYGZsq{} = \PYGZsq{}shell\PYGZsq{}

LOG     [U=1000,P=1]       main()                                    USER=gmk, IMAGE=\PYGZsq{}ubuntu.dir\PYGZsq{}, COMMAND=\PYGZsq{}shell\PYGZsq{}

INFO    [U=1000,P=1]       action\PYGZus{}shell()                            Singularity: Invoking an interactive shell within container...


DEBUG   [U=1000,P=1]       action\PYGZus{}shell()                            Exec\PYGZsq{}ing /.singularity.d/actions/shell

Singularity ubuntu.dir:\PYGZti{}\PYGZgt{} whoami

gmk

Singularity ubuntu.dir:\PYGZti{}\PYGZgt{}
\end{sphinxVerbatim}

Here you can see that the output and functionality is very similar,
but we never increased any privilege and none of the \sphinxcode{\sphinxupquote{*-suid}} program flow was
utilized. We had to use a chroot style directory container (as images
are not supported with the user namespace, but you can clearly see
that the effective UID never had to change to run this container.

\begin{sphinxadmonition}{note}{Note:}
Singularity can natively create and manage chroot style
containers just like images! The above image was created using the
command: \sphinxcode{\sphinxupquote{singularity build ubuntu.dir docker://ubuntu:latest}}
\end{sphinxadmonition}


\section{Summary}
\label{\detokenize{security:summary}}
Singularity supports multiple modes of operation to meet your security
needs. For most HPC centers, and general usage scenarios, the default
run mode is most effective and featurefull. For the security critical
implementations, the user namespace workflow maybe a better option. It
becomes a balance security and functionality (the most secure systems do
nothing).


\chapter{The Singularity Config File}
\label{\detokenize{the_singularity_config_file:the-singularity-config-file}}\label{\detokenize{the_singularity_config_file::doc}}
When Singularity is running via the SUID pathway, the configuration
\sphinxstylestrong{must} be owned by the root user otherwise Singularity will error
out. This ensures that the system administrators have direct say as to
what functions the users can utilize when running as root. If
Singularity is installed as a non-root user, the SUID components are
not installed, and the configuration file can be owned by the user
(but again, this will limit functionality).
The Configuration file can be found at \sphinxcode{\sphinxupquote{\$SYSCONFDIR/singularity/singularity.conf}}. The template in the
repository is located at \sphinxcode{\sphinxupquote{etc/singularity.conf}}. It is generally self documenting but there
are several things to pay special attention to:


\section{Parameters}
\label{\detokenize{the_singularity_config_file:parameters}}

\subsection{ALLOW SETUID (boolean, default=’yes’)}
\label{\detokenize{the_singularity_config_file:allow-setuid-boolean-default-yes}}
This parameter toggles the global ability to execute the SETUID (SUID)
portion of the code if it exists. As mentioned earlier, if the SUID
features are disabled, various Singularity features will not function
(e.g. mounting of the Singularity image file format).
You can however disable SUID support \sphinxstylestrong{iff} (if and only if) you do
not need to use the default Singularity image file format and if your
kernel supports user namespaces and you choose to use user namespaces.

\begin{sphinxadmonition}{note}{Note:}
As of the time of this writing, the user namespace is rather
buggy
\end{sphinxadmonition}


\subsection{ALLOW PID NS (boolean, default=’yes’)}
\label{\detokenize{the_singularity_config_file:allow-pid-ns-boolean-default-yes}}
While the PID namespace is a neat feature, it does not have much
practical usage in an HPC context so it is recommended to disable this
if you are running on an HPC system where a resource manager is
involved as it has been known to cause confusion on some kernels with
enforcement of user limits.
Even if the PID namespace is enabled by the system administrator here,
it is not implemented by default when running containers. The user
will have to specify they wish to implement un-sharing of the PID
namespace as it must fork a child process.


\subsection{ENABLE OVERLAY (boolean, default=’no’)}
\label{\detokenize{the_singularity_config_file:enable-overlay-boolean-default-no}}
The overlay file system creates a writable substrate to create bind
points if necessary. This feature is very useful when implementing bind
points within containers where the bind point may not already exist so
it helps with portability of containers. Enabling this option has been
known to cause some kernels to panic as this feature maybe present
within a kernel, but has not proved to be stable as of the time of this
writing (e.g. the Red Hat 7.2 kernel).


\subsection{CONFIG PASSWD, GROUP, RESOLV\_CONF (boolean, default=’yes’)}
\label{\detokenize{the_singularity_config_file:config-passwd-group-resolv-conf-boolean-default-yes}}
All of these options essentially do the same thing for different files
within the container. This feature updates the described file (\sphinxcode{\sphinxupquote{/etc/passwd}}, \sphinxcode{\sphinxupquote{/etc/group}} , and \sphinxcode{\sphinxupquote{/etc/resolv.conf}}
respectively) to be updated dynamically as the container is executed. It
uses binds and modifies temporary files such that the original files are
not manipulated.


\subsection{MOUNT PROC,SYS,DEV,HOME,TMP (boolean, default=’yes’)}
\label{\detokenize{the_singularity_config_file:mount-proc-sys-dev-home-tmp-boolean-default-yes}}
These configuration options control the mounting of these file systems
within the container and of course can be overridden by the system
administrator (e.g. the system admin decides not to include the /dev
tree inside the container). In most useful cases, these are all best to
leave enabled.


\subsection{MOUNT HOSTFS (boolean, default=’no’)}
\label{\detokenize{the_singularity_config_file:mount-hostfs-boolean-default-no}}
This feature will parse the host’s mounted file systems and attempt to
replicate all mount points within the container. This maybe a desirable
feature for the lazy, but it is generally better to statically define
what bind points you wish to encapsulate within the container by hand
(using the below “bind path” feature).


\subsection{BIND PATH (string)}
\label{\detokenize{the_singularity_config_file:bind-path-string}}
With this configuration directive, you can specify any number of bind
points that you want to extend from the host system into the
container. Bind points on the host file system must be either real
files or directories (no special files supported at this time). If the
overlayFS is not supported on your host, or if \sphinxcode{\sphinxupquote{enable overlay = no}} in this configuration
file, a bind point must exist for the file or directory within the
container.
The syntax for this consists of a bind path source and an optional
bind path destination separated by a colon. If no bind path
destination is specified, the bind path source is used also as the
destination.


\subsection{USER BIND CONTROL (boolean, default=’yes’)}
\label{\detokenize{the_singularity_config_file:user-bind-control-boolean-default-yes}}
In addition to the system bind points as specified within this
configuration file, you may also allow users to define their own bind
points inside the container. This feature is used via multiple command
line arguments (e.g. \sphinxcode{\sphinxupquote{-{-}bind}}, \sphinxcode{\sphinxupquote{-{-}scratch}} , and \sphinxcode{\sphinxupquote{-{-}home}}) so disabling user bind control will
also disable those command line options.
Singularity will automatically disable this feature if the host does
not support the prctl option \sphinxcode{\sphinxupquote{PR\_SET\_NO\_NEW\_PRIVS}}. In addition, \sphinxcode{\sphinxupquote{enable overlay}} must be set to \sphinxcode{\sphinxupquote{yes}} and the
host system must support overlayFS (generally kernel versions 3.18 and
later) for users to bind host directories to bind points that do not
already exist in the container.


\subsection{AUTOFS BUG PATH (string)}
\label{\detokenize{the_singularity_config_file:autofs-bug-path-string}}
With some versions of autofs, Singularity will fail to run with a “Too
many levels of symbolic links” error. This error happens by way of a
user requested bind (done with -B/\textendash{}bind) or one specified via the
configuration file. To handle this, you will want to specify those
paths using this directive. For example:

\fvset{hllines={, ,}}%
\begin{sphinxVerbatim}[commandchars=\\\{\}]
autofs bug path = /share/PI
\end{sphinxVerbatim}


\section{Logging}
\label{\detokenize{the_singularity_config_file:logging}}
In order to facilitate monitoring and auditing, Singularity will
syslog() every action and error that takes place to the \sphinxcode{\sphinxupquote{LOCAL0}} syslog facility.
You can define what to do with those logs in your syslog configuration.


\section{Loop Devices}
\label{\detokenize{the_singularity_config_file:loop-devices}}
Singularity images have \sphinxcode{\sphinxupquote{ext3}} file systems embedded within them, and thus to
mount them, we need to convert the raw file system image (with
variable offset) to a block device. To do this, Singularity utilizes
the \sphinxcode{\sphinxupquote{/dev/loop*}} block devices on the host system and manages the devices
programmatically within Singularity itself. Singularity also uses the \sphinxcode{\sphinxupquote{LO\_FLAGS\_AUTOCLEAR}}
loop device \sphinxcode{\sphinxupquote{ioctl()}} flag which tells the kernel to automatically free the loop
device when there are no more open file descriptors to the device
itself.
Earlier versions of Singularity managed the loop devices via a
background watchdog process, but since version 2.2 we leverage the \sphinxcode{\sphinxupquote{LO\_FLAGS\_AUTOCLEAR}}
functionality and we forego the watchdog process. Unfortunately, this
means that some older Linux distributions are no longer supported
(e.g. RHEL \textless{}= 5).
Given that loop devices are consumable (there are a limited number of
them on a system), Singularity attempts to be smart in how loop
devices are allocated. For example, if a given user executes a
specific container it will bind that image to the next available loop
device automatically. If that same user executes another command on
the same container, it will use the loop device that has already been
allocated instead of binding to another loop device. Most Linux
distributions only support 8 loop devices by default, so if you find
that you have a lot of different users running Singularity containers,
you may need to increase the number of loop devices that your system
supports by doing the following:
Edit or create the file \sphinxcode{\sphinxupquote{/etc/modprobe.d/loop.conf}} and add the following line:

\fvset{hllines={, ,}}%
\begin{sphinxVerbatim}[commandchars=\\\{\}]
options loop max\PYGZus{}loop=128
\end{sphinxVerbatim}

After making this change, you should be able to reboot your system or
unload/reload the loop device as root using the following commands:

\fvset{hllines={, ,}}%
\begin{sphinxVerbatim}[commandchars=\\\{\}]
\PYGZsh{} modprobe \PYGZhy{}r loop

\PYGZsh{} modprobe loop
\end{sphinxVerbatim}


\chapter{Container Checks}
\label{\detokenize{container_checks:container-checks}}\label{\detokenize{container_checks::doc}}
New to Singularity 2.4 is the ability to, on demand, run container
“checks,” which can be anything from a filter for sensitive information,
to an analysis of content on the filesystem. Checks are installed with
Singularity, managed by the administrator, and \sphinxhref{https://singularity-userdoc.readthedocs.io/en/latest/container\_checks.html}{available to the
user}.


\section{What is a check?}
\label{\detokenize{container_checks:what-is-a-check}}
Broadly, a check is a script that is run over a mounted filesystem,
primary with the purpose of checking for some security issue. This
process is tightly controlled, meaning that the script names in the
\sphinxhref{https://github.com/singularityware/singularity/tree/development/libexec/helpers/checks}{checks}
folder are hard coded into the script
\sphinxhref{https://github.com/singularityware/singularity/blob/development/libexec/helpers/check.sh}{check.sh}.
The flow of checks is the following:
\begin{itemize}
\item {} 
the user calls \sphinxcode{\sphinxupquote{singularity check container.img}} to invoke
\sphinxhref{https://github.com/singularityware/singularity/blob/development/libexec/cli/check.exec}{check.exec}

\item {} 
specification of {\color{red}\bfseries{}{}`{}`}\textendash{}low{}`{}`(3), {\color{red}\bfseries{}{}`{}`}\textendash{}med{}`{}`(2), or {\color{red}\bfseries{}{}`{}`}\textendash{}high{}`{}`(1) sets the level to perform. The
level is a filter, meaning that a level of 3 will include 3,2,1, and
a level of 1 (high) will only call checks of high priority.

\item {} 
specification of \sphinxcode{\sphinxupquote{-t/-{-}tag}} will allow the user (or execution script) to specify
a kind of check. This is primarily to allow for extending the checks
to do other types of things. For example, for this initial batch,
these are all considered \sphinxcode{\sphinxupquote{default}} checks. The
\sphinxhref{https://github.com/singularityware/singularity/blob/development/libexec/cli/check.help}{check.help}
displays examples of how the user specifies a tag:

\end{itemize}

\fvset{hllines={, ,}}%
\begin{sphinxVerbatim}[commandchars=\\\{\}]
\PYGZsh{} Perform all default checks, these are the same

\PYGZdl{} singularity check ubuntu.img

\PYGZdl{} singularity check \PYGZhy{}\PYGZhy{}tag default ubuntu.img


\PYGZsh{} Perform checks with tag \PYGZdq{}clean\PYGZdq{}

\PYGZdl{} singularity check \PYGZhy{}\PYGZhy{}tag clean ubuntu.img
\end{sphinxVerbatim}


\subsection{Adding a Check}
\label{\detokenize{container_checks:adding-a-check}}
A check should be a bash (or other) script that will perform some
action. The following is required:
\sphinxstylestrong{Relative to SINGULARITY\_ROOTFS} The script must perform check
actions relative to \sphinxcode{\sphinxupquote{SINGULARITY\_ROOTFS}}. For example, in python you might change
directory to this location:

\fvset{hllines={, ,}}%
\begin{sphinxVerbatim}[commandchars=\\\{\}]
import os

base = os.environ[\PYGZdq{}SINGULARITY\PYGZus{}ROOTFS\PYGZdq{}]

os.chdir(base)
\end{sphinxVerbatim}

or do the same in bash:

\fvset{hllines={, ,}}%
\begin{sphinxVerbatim}[commandchars=\\\{\}]
cd \PYGZdl{}SINGULARITY\PYGZus{}ROOTFS

ls \PYGZdl{}SINGULARITY\PYGZus{}ROOTFS/var
\end{sphinxVerbatim}

Since we are doing a mount, all checks must be static relative to this
base, otherwise you are likely checking the host system.

\sphinxstylestrong{Verbose} The script should indicate any warning/message to the user
if the check is found to have failed. If pass, the check’s name and
status will be printed, with any relevant information. For more
thorough checking, you might want to give more verbose output.

\sphinxstylestrong{Return Code} The script return code of “success” is defined in
\sphinxhref{https://github.com/singularityware/singularity/blob/development/libexec/helpers/check.sh}{check.sh}, and other return
codes are considered not success. When a non success return code is
found, the rest of the checks continue running, and no action is
taken. We might want to give some admin an ability to specify a check,
a level, and prevent continuation of the build/bootstrap given a fail.
\sphinxstylestrong{Check.sh} The script level, path, and tags should be added to
\sphinxhref{https://github.com/singularityware/singularity/blob/development/libexec/helpers/check.sh}{check.sh} in the following
\begin{quote}

format:
\end{quote}

\fvset{hllines={, ,}}%
\begin{sphinxVerbatim}[commandchars=\\\{\}]
\PYGZsh{}\PYGZsh{}\PYGZsh{}\PYGZsh{}\PYGZsh{}\PYGZsh{}\PYGZsh{}\PYGZsh{}\PYGZsh{}\PYGZsh{}\PYGZsh{}\PYGZsh{}\PYGZsh{}\PYGZsh{}\PYGZsh{}\PYGZsh{}\PYGZsh{}\PYGZsh{}\PYGZsh{}\PYGZsh{}\PYGZsh{}\PYGZsh{}\PYGZsh{}\PYGZsh{}\PYGZsh{}\PYGZsh{}\PYGZsh{}\PYGZsh{}\PYGZsh{}\PYGZsh{}\PYGZsh{}\PYGZsh{}\PYGZsh{}\PYGZsh{}\PYGZsh{}\PYGZsh{}\PYGZsh{}\PYGZsh{}\PYGZsh{}\PYGZsh{}\PYGZsh{}\PYGZsh{}\PYGZsh{}\PYGZsh{}\PYGZsh{}\PYGZsh{}\PYGZsh{}\PYGZsh{}\PYGZsh{}\PYGZsh{}\PYGZsh{}\PYGZsh{}\PYGZsh{}\PYGZsh{}\PYGZsh{}\PYGZsh{}\PYGZsh{}\PYGZsh{}\PYGZsh{}\PYGZsh{}\PYGZsh{}\PYGZsh{}\PYGZsh{}\PYGZsh{}\PYGZsh{}\PYGZsh{}\PYGZsh{}\PYGZsh{}\PYGZsh{}\PYGZsh{}\PYGZsh{}\PYGZsh{}\PYGZsh{}\PYGZsh{}\PYGZsh{}\PYGZsh{}\PYGZsh{}\PYGZsh{}\PYGZsh{}\PYGZsh{}\PYGZsh{}\PYGZsh{}

\PYGZsh{} CHECK SCRIPTS

\PYGZsh{}\PYGZsh{}\PYGZsh{}\PYGZsh{}\PYGZsh{}\PYGZsh{}\PYGZsh{}\PYGZsh{}\PYGZsh{}\PYGZsh{}\PYGZsh{}\PYGZsh{}\PYGZsh{}\PYGZsh{}\PYGZsh{}\PYGZsh{}\PYGZsh{}\PYGZsh{}\PYGZsh{}\PYGZsh{}\PYGZsh{}\PYGZsh{}\PYGZsh{}\PYGZsh{}\PYGZsh{}\PYGZsh{}\PYGZsh{}\PYGZsh{}\PYGZsh{}\PYGZsh{}\PYGZsh{}\PYGZsh{}\PYGZsh{}\PYGZsh{}\PYGZsh{}\PYGZsh{}\PYGZsh{}\PYGZsh{}\PYGZsh{}\PYGZsh{}\PYGZsh{}\PYGZsh{}\PYGZsh{}\PYGZsh{}\PYGZsh{}\PYGZsh{}\PYGZsh{}\PYGZsh{}\PYGZsh{}\PYGZsh{}\PYGZsh{}\PYGZsh{}\PYGZsh{}\PYGZsh{}\PYGZsh{}\PYGZsh{}\PYGZsh{}\PYGZsh{}\PYGZsh{}\PYGZsh{}\PYGZsh{}\PYGZsh{}\PYGZsh{}\PYGZsh{}\PYGZsh{}\PYGZsh{}\PYGZsh{}\PYGZsh{}\PYGZsh{}\PYGZsh{}\PYGZsh{}\PYGZsh{}\PYGZsh{}\PYGZsh{}\PYGZsh{}\PYGZsh{}\PYGZsh{}\PYGZsh{}\PYGZsh{}\PYGZsh{}\PYGZsh{}\PYGZsh{}


\PYGZsh{}        [SUCCESS] [LEVEL]  [SCRIPT]                                                                         [TAGS]

execute\PYGZus{}check    0    HIGH  \PYGZdq{}bash \PYGZdl{}SINGULARITY\PYGZus{}libexecdir/singularity/helpers/checks/1\PYGZhy{}hello\PYGZhy{}world.sh\PYGZdq{}       security

execute\PYGZus{}check    0     LOW  \PYGZdq{}python \PYGZdl{}SINGULARITY\PYGZus{}libexecdir/singularity/helpers/checks/2\PYGZhy{}cache\PYGZhy{}content.py\PYGZdq{}   clean

execute\PYGZus{}check    0    HIGH  \PYGZdq{}python \PYGZdl{}SINGULARITY\PYGZus{}libexecdir/singularity/helpers/checks/3\PYGZhy{}cve.py\PYGZdq{}             security
\end{sphinxVerbatim}

The function \sphinxcode{\sphinxupquote{execute\_check}} will compare the level (\sphinxcode{\sphinxupquote{{[}LEVEL{]}}}) with the user specified (or
default) \sphinxcode{\sphinxupquote{SINGULARITY\_CHECKLEVEL}} and execute the check only given it is under the specified
threshold, and (not yet implemented) has the relevant tag. The success
code is also set here with \sphinxcode{\sphinxupquote{{[}SUCCESS{]}}}. Currently, we aren’t doing anything with \sphinxcode{\sphinxupquote{{[}TAGS{]}}}
and thus perform all checks.


\section{How to tell users?}
\label{\detokenize{container_checks:how-to-tell-users}}
If you add a custom check that you want for your users to use, you
should tell them about it. Better yet, \sphinxhref{https://github.com/singularityware/singularity/issues}{tell
us} about it
so it can be integrated into the Singularity software for others to use.


\chapter{Troubleshooting}
\label{\detokenize{troubleshooting:troubleshooting}}\label{\detokenize{troubleshooting::doc}}
This section will help you debug (from the system administrator’s
perspective) Singularity.


\section{Not installed correctly, or installed to a non-compatible location}
\label{\detokenize{troubleshooting:not-installed-correctly-or-installed-to-a-non-compatible-location}}
Singularity must be installed by root into a location that allows for
\sphinxcode{\sphinxupquote{SUID}} programs to be executed (as described above in the installation
section of this manual). If you fail to do that, you may have user’s
reporting one of the following error conditions:

\fvset{hllines={, ,}}%
\begin{sphinxVerbatim}[commandchars=\\\{\}]
ERROR  : Singularity must be executed in privileged mode to use images

ABORT  : Retval = 255
\end{sphinxVerbatim}

\fvset{hllines={, ,}}%
\begin{sphinxVerbatim}[commandchars=\\\{\}]
ERROR  : User namespace not supported, and program not running privileged.

ABORT  : Retval = 255
\end{sphinxVerbatim}

\fvset{hllines={, ,}}%
\begin{sphinxVerbatim}[commandchars=\\\{\}]
ABORT  : This program must be SUID root

ABORT  : Retval = 255
\end{sphinxVerbatim}

If one of these errors is reported, it is best to check the installation
of Singularity and ensure that it was properly installed by the root
user onto a local file system.


\chapter{Installation Environments}
\label{\detokenize{installation_environments:installation-environments}}\label{\detokenize{installation_environments::doc}}

\section{Singularity on HPC}
\label{\detokenize{installation_environments:singularity-on-hpc}}
One of the architecturally defined features in Singularity is that it
can execute containers like they are native programs or scripts on a
host computer. As a result, integration with schedulers is simple and
runs exactly as you would expect. All standard input, output, error,
pipes, IPC, and other communication pathways that locally running
programs employ are synchronized with the applications running locally
within the container.
Additionally, because Singularity is not emulating a full hardware
level virtualization paradigm, there is no need to separate out any
sandboxed networks or file systems because there is no concept of
user-escalation within a container. Users can run Singularity
containers just as they run any other program on the HPC resource.


\subsection{Workflows}
\label{\detokenize{installation_environments:workflows}}
We are in the process of developing Singularity Hub, which will allow
for generation of workflows using Singularity containers in an online
interface, and easy deployment on standard research clusters (e.g.,
SLURM, SGE). Currently, the Singularity core software is installed on
the following research clusters, meaning you can run Singularity
containers as part of your jobs:
\begin{itemize}
\item {} 
The \sphinxhref{http://sherlock.stanford.edu/}{Sherlock cluster} at \sphinxhref{https://srcc.stanford.edu/}{Stanford
University}

\item {} 
\sphinxhref{https://www.xsede.org/news/-/news/item/7624}{SDSC Comet and
Gordon} (XSEDE)

\item {} 
\sphinxhref{http://docs.massive.org.au/index.html}{MASSIVE M1 M2 and M3}
(Monash University and Australian National Merit Allocation Scheme)

\end{itemize}


\subsubsection{Integration with MPI}
\label{\detokenize{installation_environments:integration-with-mpi}}
Another result of the Singularity architecture is the ability to
properly integrate with the Message Passing Interface (MPI). Work has
already been done for out of the box compatibility with Open MPI (both
in Open MPI v2.1.x as well as part of Singularity). The Open
MPI/Singularity workflow works as follows:
\begin{enumerate}
\item {} 
mpirun is called by the resource manager or the user directly from a
shell

\item {} 
Open MPI then calls the process management daemon (ORTED)

\item {} 
The ORTED process launches the Singularity container requested by the
mpirun command

\item {} 
Singularity builds the container and namespace environment

\item {} 
Singularity then launches the MPI application within the container

\item {} 
The MPI application launches and loads the Open MPI libraries

\item {} 
The Open MPI libraries connect back to the ORTED process via the
Process Management Interface (PMI)

\item {} 
At this point the processes within the container run as they would
normally directly on the host.

\end{enumerate}

This entire process happens behind the scenes, and from the user’s
perspective running via MPI is as simple as just calling mpirun on the
host as they would normally.
Below are example snippets of building and installing OpenMPI into a
container and then running an example MPI program through Singularity.


\subsubsection{Tutorials}
\label{\detokenize{installation_environments:tutorials}}
{\hyperref[\detokenize{appendix:using-host-libraries-gpu-drivers-and-openmpi-btls}]{\sphinxcrossref{\DUrole{std,std-ref}{Using Host libraries: GPU drivers and OpenMPI BTLs}}}}


\subsubsection{MPI Development Example}
\label{\detokenize{installation_environments:mpi-development-example}}
\sphinxstylestrong{What are supported Open MPI Version(s)?} To achieve proper
container’ized Open MPI support, you should use Open MPI version 2.1.
There are however three caveats:
\begin{enumerate}
\item {} 
Open MPI 1.10.x may work but we expect you will need exactly matching
version of PMI and Open MPI on both host and container (the 2.1
series should relax this requirement)

\item {} 
Open MPI 2.1.0 has a bug affecting compilation of libraries for some
interfaces (particularly Mellanox interfaces using libmxm are known
to fail). If your in this situation you should use the master branch
of Open MPI rather than the release.

\item {} 
Using Open MPI 2.1 does not magically allow your container to connect
to networking fabric libraries in the host. If your cluster has, for
example, an infiniband network you still need to install OFED
libraries into the container. Alternatively you could bind mount both
Open MPI and networking libraries into the container, but this could
run afoul of glib compatibility issues (its generally OK if the
container glibc is more recent than the host, but not the other way
around)

\end{enumerate}


\subsubsection{Code Example using Open MPI 2.1.0 Stable}
\label{\detokenize{installation_environments:code-example-using-open-mpi-2-1-0-stable}}
\fvset{hllines={, ,}}%
\begin{sphinxVerbatim}[commandchars=\\\{\}]
\PYGZdl{} \PYGZsh{} Include the appropriate development tools into the container (notice we are calling

\PYGZdl{} \PYGZsh{} singularity as root and the container is writable)

\PYGZdl{} sudo singularity exec \PYGZhy{}w /tmp/Centos\PYGZhy{}7.img yum groupinstall \PYGZdq{}Development Tools\PYGZdq{}

\PYGZdl{}

\PYGZdl{} \PYGZsh{} Obtain the development version of Open MPI

\PYGZdl{} wget https://www.open\PYGZhy{}mpi.org/software/ompi/v2.1/downloads/openmpi\PYGZhy{}2.1.0.tar.bz2

\PYGZdl{} tar jtf openmpi\PYGZhy{}2.1.0.tar.bz2

\PYGZdl{} cd openmpi\PYGZhy{}2.1.0

\PYGZdl{}

\PYGZdl{} singularity exec /tmp/Centos\PYGZhy{}7.img ./configure \PYGZhy{}\PYGZhy{}prefix=/usr/local

\PYGZdl{} singularity exec /tmp/Centos\PYGZhy{}7.img make

\PYGZdl{}

\PYGZdl{} \PYGZsh{} Install OpenMPI into the container (notice now running as root and container is writable)

\PYGZdl{} sudo singularity exec \PYGZhy{}w \PYGZhy{}B /home /tmp/Centos\PYGZhy{}7.img make install

\PYGZdl{}

\PYGZdl{} \PYGZsh{} Build the OpenMPI ring example and place the binary in this directory

\PYGZdl{} singularity exec /tmp/Centos\PYGZhy{}7.img mpicc examples/ring\PYGZus{}c.c \PYGZhy{}o ring

\PYGZdl{}

\PYGZdl{} \PYGZsh{} Install the MPI binary into the container at /usr/bin/ring

\PYGZdl{} sudo singularity copy /tmp/Centos\PYGZhy{}7.img ./ring /usr/bin/

\PYGZdl{}

\PYGZdl{} \PYGZsh{} Run the MPI program within the container by calling the MPIRUN on the host

\PYGZdl{} mpirun \PYGZhy{}np 20 singularity exec /tmp/Centos\PYGZhy{}7.img /usr/bin/ring
\end{sphinxVerbatim}


\subsubsection{Code Example using Open MPI git master}
\label{\detokenize{installation_environments:code-example-using-open-mpi-git-master}}
The previous example (using the Open MPI 2.1.0 stable release) should
work fine on most hardware but if you have an issue, try running the
example below (using the Open MPI Master branch):

\fvset{hllines={, ,}}%
\begin{sphinxVerbatim}[commandchars=\\\{\}]
\PYGZdl{} \PYGZsh{} Include the appropriate development tools into the container (notice we are calling

\PYGZdl{} \PYGZsh{} singularity as root and the container is writable)

\PYGZdl{} sudo singularity exec \PYGZhy{}w /tmp/Centos\PYGZhy{}7.img yum groupinstall \PYGZdq{}Development Tools\PYGZdq{}

\PYGZdl{}

\PYGZdl{} \PYGZsh{} Clone the OpenMPI GitHub master branch in current directory (on host)

\PYGZdl{} git clone https://github.com/open\PYGZhy{}mpi/ompi.git

\PYGZdl{} cd ompi

\PYGZdl{}

\PYGZdl{} \PYGZsh{} Build OpenMPI in the working directory, using the tool chain within the container

\PYGZdl{} singularity exec /tmp/Centos\PYGZhy{}7.img ./autogen.pl

\PYGZdl{} singularity exec /tmp/Centos\PYGZhy{}7.img ./configure \PYGZhy{}\PYGZhy{}prefix=/usr/local

\PYGZdl{} singularity exec /tmp/Centos\PYGZhy{}7.img make

\PYGZdl{}

\PYGZdl{} \PYGZsh{} Install OpenMPI into the container (notice now running as root and container is writable)

\PYGZdl{} sudo singularity exec \PYGZhy{}w \PYGZhy{}B /home /tmp/Centos\PYGZhy{}7.img make install

\PYGZdl{}

\PYGZdl{} \PYGZsh{} Build the OpenMPI ring example and place the binary in this directory

\PYGZdl{} singularity exec /tmp/Centos\PYGZhy{}7.img mpicc examples/ring\PYGZus{}c.c \PYGZhy{}o ring

\PYGZdl{}

\PYGZdl{} \PYGZsh{} Install the MPI binary into the container at /usr/bin/ring

\PYGZdl{} sudo singularity copy /tmp/Centos\PYGZhy{}7.img ./ring /usr/bin/

\PYGZdl{}

\PYGZdl{} \PYGZsh{} Run the MPI program within the container by calling the MPIRUN on the host

\PYGZdl{} mpirun \PYGZhy{}np 20 singularity exec /tmp/Centos\PYGZhy{}7.img /usr/bin/ring



Process 0 sending 10 to 1, tag 201 (20 processes in ring)

Process 0 sent to 1

Process 0 decremented value: 9

Process 0 decremented value: 8

Process 0 decremented value: 7

Process 0 decremented value: 6

Process 0 decremented value: 5

Process 0 decremented value: 4

Process 0 decremented value: 3

Process 0 decremented value: 2

Process 0 decremented value: 1

Process 0 decremented value: 0

Process 0 exiting

Process 1 exiting

Process 2 exiting

Process 3 exiting

Process 4 exiting

Process 5 exiting

Process 6 exiting

Process 7 exiting

Process 8 exiting

Process 9 exiting

Process 10 exiting

Process 11 exiting

Process 12 exiting

Process 13 exiting

Process 14 exiting

Process 15 exiting

Process 16 exiting

Process 17 exiting

Process 18 exiting

Process 19 exiting
\end{sphinxVerbatim}


\section{Image Environment}
\label{\detokenize{installation_environments:image-environment}}

\subsection{Directory access}
\label{\detokenize{installation_environments:directory-access}}
By default Singularity tries to create a seamless user experience
between the host and the container. To do this, Singularity makes
various locations accessible within the container automatically. For
example, the user’s home directory is always bound into the container as
is /tmp and /var/tmp. Additionally your current working directory
(cwd/pwd) is also bound into the container iff it is not an operating
system directory or already accessible via another mount. For almost all
cases, this will work flawlessly as follows:

\fvset{hllines={, ,}}%
\begin{sphinxVerbatim}[commandchars=\\\{\}]
\PYGZdl{} pwd

/home/gmk/demo

\PYGZdl{} singularity shell container.img

Singularity/container.img\PYGZgt{} pwd

/home/gmk/demo

Singularity/container.img\PYGZgt{} ls \PYGZhy{}l debian.def

\PYGZhy{}rw\PYGZhy{}rw\PYGZhy{}r\PYGZhy{}\PYGZhy{}. 1 gmk gmk 125 May 28 10:35 debian.def

Singularity/container.img\PYGZgt{} exit

\PYGZdl{}
\end{sphinxVerbatim}

For directory binds to function properly, there must be an existing
target endpoint within the container (just like a mount point). This
means that if your home directory exists in a non-standard base
directory like “/foobar/username” then the base directory “/foobar”
must already exist within the container.
Singularity will not create these base directories! You must enter the
container with the option \sphinxcode{\sphinxupquote{-{-}writable}} being set, and create the directory
manually.


\subsubsection{Current Working Directory}
\label{\detokenize{installation_environments:current-working-directory}}
Singularity will try to replicate your current working directory within
the container. Sometimes this is straight forward and possible, other
times it is not (e.g. if the base dir of your current working directory
does not exist). In that case, Singularity will retain the file
descriptor to your current directory and change you back to it. If you
do a ‘pwd’ within the container, you may see some weird things. For
example:

\fvset{hllines={, ,}}%
\begin{sphinxVerbatim}[commandchars=\\\{\}]
\PYGZdl{} pwd

/foobar

\PYGZdl{} ls \PYGZhy{}l

total 0

\PYGZhy{}rw\PYGZhy{}r\PYGZhy{}\PYGZhy{}r\PYGZhy{}\PYGZhy{}. 1 root root 0 Jun  1 11:32 mooooo

\PYGZdl{} singularity shell \PYGZti{}/demo/container.img

WARNING: CWD bind directory not present: /foobar

Singularity/container.img\PYGZgt{} pwd

(unreachable)/foobar

Singularity/container.img\PYGZgt{} ls \PYGZhy{}l

total 0

\PYGZhy{}rw\PYGZhy{}r\PYGZhy{}\PYGZhy{}r\PYGZhy{}\PYGZhy{}. 1 root root 0 Jun  1 18:32 mooooo

Singularity/container.img\PYGZgt{} exit

\PYGZdl{}
\end{sphinxVerbatim}

But notice how even though the directory location is not resolvable, the
directory contents are available.


\subsection{Standard IO and pipes}
\label{\detokenize{installation_environments:standard-io-and-pipes}}
Singularity automatically sends and receives all standard IO from the
host to the applications within the container to facilitate expected
behavior from the interaction between the host and the container. For
example:

\fvset{hllines={, ,}}%
\begin{sphinxVerbatim}[commandchars=\\\{\}]
\PYGZdl{} cat debian.def \textbar{} singularity exec container.img grep \PYGZsq{}MirrorURL\PYGZsq{}

MirrorURL \PYGZdq{}http://ftp.us.debian.org/debian/\PYGZdq{}

\PYGZdl{}

Making changes to the container (writable)

By default, containers are accessed as read only. This is both to enable parallel container execution (e.g. MPI). To enter a container using exec, run, or shell you must pass the \PYGZhy{}\PYGZhy{}writable flag in order to open the image as read/writable.
\end{sphinxVerbatim}


\subsection{Containing the container}
\label{\detokenize{installation_environments:containing-the-container}}
By providing the argument \sphinxcode{\sphinxupquote{-{-}contain}} to \sphinxcode{\sphinxupquote{exec}}, \sphinxcode{\sphinxupquote{run}} or \sphinxcode{\sphinxupquote{shell}} you will find that shared directories
are no longer shared. For example, the user’s home directory is
writable, but it is non-persistent between non-overlapping runs.


\section{License}
\label{\detokenize{installation_environments:license}}
\fvset{hllines={, ,}}%
\begin{sphinxVerbatim}[commandchars=\\\{\}]
Redistribution and use in source and binary forms, with or without

modification, are permitted provided that the following conditions are met:


(1) Redistributions of source code must retain the above copyright notice,

this list of conditions and the following disclaimer.


(2) Redistributions in binary form must reproduce the above copyright notice,

this list of conditions and the following disclaimer in the documentation

and/or other materials provided with the distribution.


(3) Neither the name of the University of California, Lawrence Berkeley

National Laboratory, U.S. Dept. of Energy nor the names of its contributors

may be used to endorse or promote products derived from this software without

specific prior written permission.


THIS SOFTWARE IS PROVIDED BY THE COPYRIGHT HOLDERS AND CONTRIBUTORS \PYGZdq{}AS IS\PYGZdq{}

AND ANY EXPRESS OR IMPLIED WARRANTIES, INCLUDING, BUT NOT LIMITED TO, THE

IMPLIED WARRANTIES OF MERCHANTABILITY AND FITNESS FOR A PARTICULAR PURPOSE ARE

DISCLAIMED. IN NO EVENT SHALL THE COPYRIGHT OWNER OR CONTRIBUTORS BE LIABLE

FOR ANY DIRECT, INDIRECT, INCIDENTAL, SPECIAL, EXEMPLARY, OR CONSEQUENTIAL

DAMAGES (INCLUDING, BUT NOT LIMITED TO, PROCUREMENT OF SUBSTITUTE GOODS OR

SERVICES; LOSS OF USE, DATA, OR PROFITS; OR BUSINESS INTERRUPTION) HOWEVER

CAUSED AND ON ANY THEORY OF LIABILITY, WHETHER IN CONTRACT, STRICT LIABILITY,

OR TORT (INCLUDING NEGLIGENCE OR OTHERWISE) ARISING IN ANY WAY OUT OF THE USE

OF THIS SOFTWARE, EVEN IF ADVISED OF THE POSSIBILITY OF SUCH DAMAGE.


You are under no obligation whatsoever to provide any bug fixes, patches, or

upgrades to the features, functionality or performance of the source code

(\PYGZdq{}Enhancements\PYGZdq{}) to anyone; however, if you choose to make your Enhancements

available either publicly, or directly to Lawrence Berkeley National

Laboratory, without imposing a separate written license agreement for such

Enhancements, then you hereby grant the following license: a  non\PYGZhy{}exclusive,

royalty\PYGZhy{}free perpetual license to install, use, modify, prepare derivative

works, incorporate into other computer software, distribute, and sublicense

such enhancements or derivative works thereof, in binary and source code form.


If you have questions about your rights to use or distribute this software,

please contact Berkeley Lab\PYGZsq{}s Innovation \PYGZam{} Partnerships Office at

IPO@lbl.gov.


NOTICE.  This Software was developed under funding from the U.S. Department of

Energy and the U.S. Government consequently retains certain rights. As such,

the U.S. Government has been granted for itself and others acting on its

behalf a paid\PYGZhy{}up, nonexclusive, irrevocable, worldwide license in the Software

to reproduce, distribute copies to the public, prepare derivative works, and

perform publicly and display publicly, and to permit other to do so.
\end{sphinxVerbatim}


\subsection{In layman terms…}
\label{\detokenize{installation_environments:in-layman-terms}}
In addition to the (already widely used and very free open source)
standard BSD 3 clause license, there is also wording specific to
contributors which ensures that we have permission to release,
distribute and include a particular contribution, enhancement, or fix as
part of Singularity proper. For example any contributions submitted will
have the standard BSD 3 clause terms (unless specifically and otherwise
stated) and that the contribution is comprised of original new code that
the contributor has authority to contribute.


\chapter{Appendix}
\label{\detokenize{appendix:appendix}}\label{\detokenize{appendix::doc}}

\section{Using Host libraries: GPU drivers and OpenMPI BTLs}
\label{\detokenize{appendix:using-host-libraries-gpu-drivers-and-openmpi-btls}}\label{\detokenize{appendix:id1}}\phantomsection\label{\detokenize{appendix:sec-tutorial-gpu-drivers-and-openmpi}}
\begin{sphinxadmonition}{note}{Note:}
\sphinxstylestrong{Much of the GPU portion of this tutorial is deprecated by the} \sphinxcode{\sphinxupquote{-{-}nv}} \sphinxstylestrong{option
that automatically binds host system driver libraries into your container at
runtime. See the} \sphinxcode{\sphinxupquote{exec}} \sphinxstylestrong{command for an example.}
\end{sphinxadmonition}

Singularity does a fantastic job of isolating you from the host so you don’t have to muck
about with \sphinxcode{\sphinxupquote{LD\_LIBRARY\_PATH}}, you just get exactly the library versions you want. However,
in some situations you need to use library versions that match host exactly. Two common ones
are NVIDIA gpu driver user-space libraries, and OpenMPI transport drivers for high performance
networking. There are many ways to solve these problems. Some people build a container and copy
the version of the libs (installed on the host) into the container.


\subsection{What We will learn today}
\label{\detokenize{appendix:what-we-will-learn-today}}
This document describes how to use a bind mount, symlinks and ldconfig so that when the host
libraries are updated the container does not need to be rebuilt.

Note this tutorial is tested with Singularity commit \sphinxhref{https://github.com/singularityware/singularity/commit/945c6ee343a1e6101e22396a90dfdb5944f442b6}{945c6ee343a1e6101e22396a90dfdb5944f442b6},
which is part of the (current) development branch, and thus it should work with version 2.3 when
that is released. The version of OpenMPI used is 2.1.0 (versions above 2.1 should work).


\subsection{Environment}
\label{\detokenize{appendix:environment}}
In our environment we run CentOS 7 hosts with:
\begin{enumerate}
\item {} 
slurm located on \sphinxcode{\sphinxupquote{/opt/slurm-\textless{}version\textgreater{}}} and the slurm user \sphinxcode{\sphinxupquote{slurm}}

\item {} 
Mellanox network cards with drivers installed to \sphinxcode{\sphinxupquote{/opt/mellanox}} ( Specifically we run a RoCEv1
network for Lustre and MPI communications)

\item {} 
NVIDIA GPUs with drivers installed to \sphinxcode{\sphinxupquote{/lib64}}

\item {} 
OpenMPI (by default) for MPI processes

\end{enumerate}


\subsection{Creating your image}
\label{\detokenize{appendix:creating-your-image}}
Since we are building an ubuntu image, it may be easier to create an ubuntu VM to create the image.
Alternatively you can follow the recipe \sphinxhref{https://singularity-admindoc.readthedocs.io/en/latest/appendix.html\#building-an-ubuntu-image-on-a-rhel-host}{here}.

Use the following def file to create the image.

\fvset{hllines={, ,}}%
\begin{sphinxVerbatim}[commandchars=\\\{\}]
Bootstrap: debootstrap

MirrorURL: http://us.archive.ubuntu.com/ubuntu/

OSVersion: xenial

Include: apt



\PYGZpc{}post

apt install \PYGZhy{}y software\PYGZhy{}properties\PYGZhy{}common

apt\PYGZhy{}add\PYGZhy{}repository \PYGZhy{}y universe

apt update

apt install \PYGZhy{}y wget

mkdir /usr/local/openmpi \textbar{}\textbar{} echo \PYGZdq{}Directory exists\PYGZdq{}

mkdir /opt/mellanox \textbar{}\textbar{} echo \PYGZdq{}Directory exists\PYGZdq{}

mkdir /all\PYGZus{}hostlibs \textbar{}\textbar{} echo \PYGZdq{}Directory exists\PYGZdq{}

mkdir /desired\PYGZus{}hostlibs \textbar{}\textbar{} echo \PYGZdq{}Directory exists\PYGZdq{}

mkdir /etc/libibverbs.d \textbar{}\textbar{} echo \PYGZdq{}Directory exists\PYGZdq{}

echo \PYGZdq{}driver mlx4\PYGZdq{} \PYGZgt{} /etc/libibverbs.d/mlx4.driver

echo \PYGZdq{}driver mlx5\PYGZdq{} \PYGZgt{} /etc/libibverbs.d/mlx5.driver

adduser slurm \textbar{}\textbar{} echo \PYGZdq{}User exists\PYGZdq{}

wget https://gist.githubusercontent.com/l1ll1/89b3f067d5b790ace6e6767be5ea2851/raw/422c8b5446c6479285cd29d1bf5be60f1b359b90/desired\PYGZus{}hostlibs.txt \PYGZhy{}O /tmp/desired\PYGZus{}hostlibs.txt

cat /tmp/desired\PYGZus{}hostlibs.txt \textbar{} xargs \PYGZhy{}I\PYGZob{}\PYGZcb{} ln \PYGZhy{}s /all\PYGZus{}hostlibs/\PYGZob{}\PYGZcb{} /desired\PYGZus{}hostlibs/\PYGZob{}\PYGZcb{}

rm /tmp/desired\PYGZus{}hostlibs.txt
\end{sphinxVerbatim}

The mysterious \sphinxcode{\sphinxupquote{wget}} line gets a list of all the libraries that the CentOS host has in \sphinxcode{\sphinxupquote{/lib64}} that we
think its safe to use in the container. Specifically these are things like nvidia drivers.

\fvset{hllines={, ,}}%
\begin{sphinxVerbatim}[commandchars=\\\{\}]
libvdpau\PYGZus{}nvidia.so

libnvidia\PYGZhy{}opencl.so.1

libnvidia\PYGZhy{}ml.so.1

libnvidia\PYGZhy{}ml.so

libnvidia\PYGZhy{}ifr.so.1

libnvidia\PYGZhy{}ifr.so

libnvidia\PYGZhy{}fbc.so.1

libnvidia\PYGZhy{}fbc.so

libnvidia\PYGZhy{}encode.so.1

libnvidia\PYGZhy{}encode.so

libnvidia\PYGZhy{}cfg.so.1

libnvidia\PYGZhy{}cfg.so

libicudata.so.50

libicudata.so

libcuda.so.1

libcuda.so

libGLX\PYGZus{}nvidia.so.0

libGLESv2\PYGZus{}nvidia.so.2

libGLESv1\PYGZus{}CM\PYGZus{}nvidia.so.1

libEGL\PYGZus{}nvidia.so.0

libibcm.a

libibcm.so

libibcm.so.1

libibcm.so.1.0.0

libibdiag\PYGZhy{}2.1.1.so

libibdiag.a

libibdiag.la

libibdiag.so

libibdiagnet\PYGZus{}plugins\PYGZus{}ifc\PYGZhy{}2.1.1.so

libibdiagnet\PYGZus{}plugins\PYGZus{}ifc.a

libibdiagnet\PYGZus{}plugins\PYGZus{}ifc.la

libibdiagnet\PYGZus{}plugins\PYGZus{}ifc.so

libibdmcom\PYGZhy{}2.1.1.so

libibdmcom.a

libibdmcom.la

libibdmcom.so

libiberty.a

libibis\PYGZhy{}2.1.1.so.3

libibis\PYGZhy{}2.1.1.so.3.0.3

libibis.a

libibis.la

libibis.so

libibmad.a

libibmad.so

libibmad.so.5

libibmad.so.5.5.0

libibnetdisc.a

libibnetdisc.so

libibnetdisc.so.5

libibnetdisc.so.5.3.0

libibsysapi\PYGZhy{}2.1.1.so

libibsysapi.a

libibsysapi.la

libibsysapi.so

libibumad.a

libibumad.so

libibumad.so.3

libibumad.so.3.1.0

libibus\PYGZhy{}1.0.so.5

libibus\PYGZhy{}1.0.so.5.0.503

libibus\PYGZhy{}qt.so.1

libibus\PYGZhy{}qt.so.1.3.0

libibverbs.a

libibverbs.so

libibverbs.so.1

libibverbs.so.1.0.0

liblustreapi.so

libmlx4\PYGZhy{}rdmav2.so

libmlx4.a

libmlx5\PYGZhy{}rdmav2.so

libmlx5.a

libnl.so.1

libnuma.so.1

libosmcomp.a

libosmcomp.so

libosmcomp.so.3

libosmcomp.so.3.0.6

libosmvendor.a

libosmvendor.so

libosmvendor.so.3

libosmvendor.so.3.0.8

libpciaccess.so.0

librdmacm.so.1

libwrap.so.0
\end{sphinxVerbatim}

Also note:
\begin{enumerate}
\item {} 
in \sphinxcode{\sphinxupquote{hostlibs.def}} we create a slurm user. Obviously if your \sphinxcode{\sphinxupquote{SlurmUser}} is different you should change this name.

\item {} 
We make directories for \sphinxcode{\sphinxupquote{/opt}} and \sphinxcode{\sphinxupquote{/usr/local/openmpi}}. We’re going to bindmount these from the host so we get
all the bits of OpenMPI and Mellanox and Slurm that we need.

\end{enumerate}


\subsection{Executing your image}
\label{\detokenize{appendix:executing-your-image}}
On our system we do:

\fvset{hllines={, ,}}%
\begin{sphinxVerbatim}[commandchars=\\\{\}]
SINGULARITYENV\PYGZus{}LD\PYGZus{}LIBRARY\PYGZus{}PATH=/usr/local/openmpi/2.1.0\PYGZhy{}gcc4/lib:/opt/munge\PYGZhy{}0.5.11/lib:/opt/slurm\PYGZhy{}16.05.4/lib:/opt/slurm\PYGZhy{}16.05.4/lib/slurm:/desired\PYGZus{}hostlibs:/opt/mellanox/mxm/lib/

export SINGULARITYENV\PYGZus{}LD\PYGZus{}LIBRARY\PYGZus{}PATH
\end{sphinxVerbatim}

then

\fvset{hllines={, ,}}%
\begin{sphinxVerbatim}[commandchars=\\\{\}]
srun  singularity exec \PYGZhy{}B /usr/local/openmpi:/usr/local/openmpi \PYGZhy{}B /opt:/opt \PYGZhy{}B /lib64:/all\PYGZus{}hostlibs hostlibs.img \PYGZlt{}path to binary\PYGZgt{}
\end{sphinxVerbatim}


\section{Building an Ubuntu image on a RHEL host}
\label{\detokenize{appendix:building-an-ubuntu-image-on-a-rhel-host}}\phantomsection\label{\detokenize{appendix:sec-building-ubuntu-rhel-host}}
This recipe describes how to build an Ubuntu image using Singularity on a RHEL compatible host.

\begin{sphinxadmonition}{note}{Note:}
This tutorial is intended for Singularity release 2.1.2,
and reflects standards for that version.
\end{sphinxadmonition}

In order to do this, you will need to first install the ‘debootstrap’ package onto your host. Then, you will create
a definition file that will describe how to build your Ubuntu image. Finally, you will build the image using the Singularity
commands ‘create’ and \sphinxcode{\sphinxupquote{bootstrap}}.


\subsection{Preparation}
\label{\detokenize{appendix:preparation}}
This recipe assumes that you have already installed Singularity on your computer. If you have not, follow the instructions here
to install. After Singularity is installed on your computer, you will need to install the ‘debootstrap’ package. The ‘debootstrap’
package is a tool that will allow you to create Debian-based distributions such as Ubuntu. In order to install ‘debootstrap’, you will
also need to install ‘epel-release’. You will need to download the appropriate RPM from the EPEL website. Make sure you download the correct
version of the RPM for your release.

\fvset{hllines={, ,}}%
\begin{sphinxVerbatim}[commandchars=\\\{\}]
\PYGZsh{} First, wget the appropriate RPM from the EPEL website (https://dl.fedoraproject.org/pub/epel/)

\PYGZsh{} In this example we used RHEL 7, so we downloaded epel\PYGZhy{}release\PYGZhy{}latest\PYGZhy{}7.noarch.rpm

\PYGZdl{} wget https://dl.fedoraproject.org/pub/epel/epel\PYGZhy{}release\PYGZhy{}latest\PYGZhy{}7.noarch.rpm


\PYGZsh{} Then, install your epel\PYGZhy{}release RPM

\PYGZdl{} sudo yum install epel\PYGZhy{}release\PYGZhy{}latest\PYGZhy{}7.noarch.rpm


\PYGZsh{} Finally, install debootstrap

\PYGZdl{} sudo yum install debootstrap
\end{sphinxVerbatim}


\subsubsection{Creating the Definition File}
\label{\detokenize{appendix:creating-the-definition-file}}
You will need to create a definition file to describe how to build your Ubuntu image. Definition files are plain text files that contain Singularity
keywords. By using certain Singularity keywords, you can specify how you want your image to be built. The extension ‘.def’ is recommended for user clarity.
Below is a definition file for a minimal Ubuntu image:

\fvset{hllines={, ,}}%
\begin{sphinxVerbatim}[commandchars=\\\{\}]
DistType \PYGZdq{}debian\PYGZdq{}

MirrorURL \PYGZdq{}http://us.archive.ubuntu.com/ubuntu/\PYGZdq{}

OSVersion \PYGZdq{}trusty\PYGZdq{}



Setup

Bootstrap



Cleanup

The following keywords were used in this definition file:
\end{sphinxVerbatim}
\begin{itemize}
\item {} 
DistType: DistType specifies the distribution type of your intended operating system. Because we are trying to build an Ubuntu image, the type “debian” was chosen.

\item {} 
MirrorURL: The MirrorURL specifies the download link for your intended operating system. The Ubuntu archive website is a great mirror link to use if you are building an Ubuntu image.

\item {} 
OSVersion: The OSVersion is used to specify which release of a Debian-based distribution you are using. In this example we chose “trusty” to specify that we wanted to build an Ubuntu
14.04 (Trusty Tahr) image.

\item {} 
Setup: Setup creates some of the base files and components for an OS and is highly recommended to be included in your definition file.

\item {} 
Bootstrap: Bootstrap will call apt-get to install the appropriate package to build your OS.

\item {} 
Cleanup: Cleanup will remove temporary files from the installation.

\end{itemize}

While this definition file is enough to create a working Ubuntu image, you may want increased customization of your image. There are several Singularity keywords that allow the user to do
things such as install packages or files. Some of these keywords are used in the example below:

\fvset{hllines={, ,}}%
\begin{sphinxVerbatim}[commandchars=\\\{\}]
DistType \PYGZdq{}debian\PYGZdq{}

MirrorURL \PYGZdq{}http://us.archive.ubuntu.com/ubuntu/\PYGZdq{}

OSVersion \PYGZdq{}trusty\PYGZdq{}


Setup

Bootstrap


InstallPkgs python

InstallPkgs wget

RunCmd wget https://bootstrap.pypa.io/get\PYGZhy{}pip.py

RunCmd python get\PYGZhy{}pip.py

RunCmd ln \PYGZhy{}s /usr/local/bin/pip /usr/bin/pip

RunCmd pip install \PYGZhy{}\PYGZhy{}upgrade https://storage.googleapis.com/tensorflow/linux/cpu/tensorflow\PYGZhy{}0.9.0\PYGZhy{}cp27\PYGZhy{}none\PYGZhy{}linux\PYGZus{}x86\PYGZus{}64.whl


Cleanup
\end{sphinxVerbatim}

Before going over exactly what image this definition file specifies, the remaining Singularity keywords should be introduced.
\begin{itemize}
\item {} 
InstallPkgs: InstallPkgs allows you to install any packages that you want on your newly created image.

\item {} 
InstallFile: InstallFile allows you to install files from your computer to the image.

\item {} 
RunCmd: RunCmd allows you to run a command from within the new image during the installation.

\item {} 
RunScript: RunScript adds a new line to the runscript invoked by the Singularity subcommand ‘run’. See the run page for more information.

\end{itemize}

Now that you are familiar with all of the Singularity keywords, we can take a closer look at the example above. As with the previous example, an Ubuntu image is created with the specified DistType,
MirrorURL, and OSVersion. However, after Setup and Bootstrap, we used the InstallPkgs keyword to install ‘python’ and ‘wget’. Then we used the RunCmd keyword to first download the pip installation wheel,
and then to install ‘pip’. Subsequently, we also used RunCmd to pip install \sphinxcode{\sphinxupquote{TensorFlow}}. Thus, we have created a definition file that will install ‘python’, ‘pip’, and ‘Tensorflow’ onto the new image.


\subsubsection{Creating your image}
\label{\detokenize{appendix:id2}}
Once you have created your definition file, you will be ready to actually create your image. You will do this by utilizing the Singularity ‘create’ and ‘bootstrap’ subcommands. The process for doing this
can be seen below:

\begin{sphinxadmonition}{note}{Note:}
We have saved our definition file as “ubuntu.def”
\end{sphinxadmonition}

\fvset{hllines={, ,}}%
\begin{sphinxVerbatim}[commandchars=\\\{\}]
\PYGZsh{} First we will create an empty image container called ubuntu.img

\PYGZdl{} sudo singularity create ubuntu.img

Creating a sparse image with a maximum size of 1024MiB...

INFO   : Using given image size of 1024

Formatting image (/sbin/mkfs.ext3)

Done. Image can be found at: ubuntu.img


\PYGZsh{} Next we will bootstrap the image with the operating system specified in our definition file

\PYGZdl{} sudo singularity bootstrap ubuntu.img ubuntu.def

W: Cannot check Release signature; keyring file not available /usr/share/keyrings/ubuntu\PYGZhy{}archive\PYGZhy{}keyring.gpg

I: Retrieving Release

I: Retrieving Packages

I: Validating Packages

I: Resolving dependencies of required packages...

I: Resolving dependencies of base packages...

I: Found additional base dependencies: gcc\PYGZhy{}4.8\PYGZhy{}base gnupg gpgv libapt\PYGZhy{}pkg4.12 libreadline6 libstdc++6 libusb\PYGZhy{}0.1\PYGZhy{}4 readline\PYGZhy{}common ubuntu\PYGZhy{}keyring

I: Checking component main on http://us.archive.ubuntu.com/ubuntu...

I: Retrieving adduser 3.113+nmu3ubuntu3

I: Validating adduser 3.113+nmu3ubuntu3

I: Retrieving apt 1.0.1ubuntu2

I: Validating apt 1.0.1ubuntu2

snip...

Downloading pip\PYGZhy{}8.1.2\PYGZhy{}py2.py3\PYGZhy{}none\PYGZhy{}any.whl (1.2MB)

100\PYGZpc{} \textbar{}\PYGZsh{}\PYGZsh{}\PYGZsh{}\PYGZsh{}\PYGZsh{}\PYGZsh{}\PYGZsh{}\PYGZsh{}\PYGZsh{}\PYGZsh{}\PYGZsh{}\PYGZsh{}\PYGZsh{}\PYGZsh{}\PYGZsh{}\PYGZsh{}\PYGZsh{}\PYGZsh{}\PYGZsh{}\PYGZsh{}\PYGZsh{}\PYGZsh{}\PYGZsh{}\PYGZsh{}\PYGZsh{}\PYGZsh{}\PYGZsh{}\PYGZsh{}\PYGZsh{}\PYGZsh{}\PYGZsh{}\PYGZsh{}\textbar{} 1.2MB 1.1MB/s

Collecting setuptools

Downloading setuptools\PYGZhy{}24.0.2\PYGZhy{}py2.py3\PYGZhy{}none\PYGZhy{}any.whl (441kB)

100\PYGZpc{} \textbar{}\PYGZsh{}\PYGZsh{}\PYGZsh{}\PYGZsh{}\PYGZsh{}\PYGZsh{}\PYGZsh{}\PYGZsh{}\PYGZsh{}\PYGZsh{}\PYGZsh{}\PYGZsh{}\PYGZsh{}\PYGZsh{}\PYGZsh{}\PYGZsh{}\PYGZsh{}\PYGZsh{}\PYGZsh{}\PYGZsh{}\PYGZsh{}\PYGZsh{}\PYGZsh{}\PYGZsh{}\PYGZsh{}\PYGZsh{}\PYGZsh{}\PYGZsh{}\PYGZsh{}\PYGZsh{}\PYGZsh{}\PYGZsh{}\textbar{} 450kB 2.7MB/s

Collecting wheel

Downloading wheel\PYGZhy{}0.29.0\PYGZhy{}py2.py3\PYGZhy{}none\PYGZhy{}any.whl (66kB)

100\PYGZpc{} \textbar{}\PYGZsh{}\PYGZsh{}\PYGZsh{}\PYGZsh{}\PYGZsh{}\PYGZsh{}\PYGZsh{}\PYGZsh{}\PYGZsh{}\PYGZsh{}\PYGZsh{}\PYGZsh{}\PYGZsh{}\PYGZsh{}\PYGZsh{}\PYGZsh{}\PYGZsh{}\PYGZsh{}\PYGZsh{}\PYGZsh{}\PYGZsh{}\PYGZsh{}\PYGZsh{}\PYGZsh{}\PYGZsh{}\PYGZsh{}\PYGZsh{}\PYGZsh{}\PYGZsh{}\PYGZsh{}\PYGZsh{}\PYGZsh{}\textbar{} 71kB 9.9MB/s

Installing collected packages: pip, setuptools, wheel

Successfully installed pip\PYGZhy{}8.1.2 setuptools\PYGZhy{}24.0.2 wheel\PYGZhy{}0.29.0

At this point, you have successfully created an Ubuntu image with \PYGZsq{}python\PYGZsq{}, \PYGZsq{}pip\PYGZsq{}, and \PYGZsq{}TensorFlow\PYGZsq{} on your RHEL computer.

Tips and Tricks

Here are some tips and tricks that you can use to create more efficient definition files:
\end{sphinxVerbatim}


\subsubsection{Use here documents with RunCmd}
\label{\detokenize{appendix:use-here-documents-with-runcmd}}
Using here documents with conjunction with RunCmd can be a great way to decrease the number of RunCmd keywords that you need to include
in your definition file. For example, we can substitute a here document into the previous example:

\fvset{hllines={, ,}}%
\begin{sphinxVerbatim}[commandchars=\\\{\}]
DistType \PYGZdq{}debian\PYGZdq{}

MirrorURL \PYGZdq{}http://us.archive.ubuntu.com/ubuntu/\PYGZdq{}

OSVersion \PYGZdq{}trusty\PYGZdq{}


Setup

Bootstrap


InstallPkgs python

InstallPkgs wget

RunCmd /bin/sh \PYGZlt{}\PYGZlt{}EOF

wget https://bootstrap.pypa.io/get\PYGZhy{}pip.py

python get\PYGZhy{}pip.py

ln \PYGZhy{}s /usr/local/bin/pip /usr/bin/pip

pip install \PYGZhy{}\PYGZhy{}upgrade https://storage.googleapis.com/tensorflow/linux/cpu/tensorflow\PYGZhy{}0.9.0\PYGZhy{}cp27\PYGZhy{}none\PYGZhy{}linux\PYGZus{}x86\PYGZus{}64.whl

EOF


Cleanup
\end{sphinxVerbatim}

As you can see, using a here document allowed us to decrease the number of RunCmd keywords from 4 to 1. This can be useful when your definition file
has a lot of RunCmd keywords and can also ease copying and pasting command line recipes from other sources.


\subsubsection{Use InstallPkgs with multiple packages}
\label{\detokenize{appendix:use-installpkgs-with-multiple-packages}}
The InstallPkgs keyword is able to install multiple packages with a single keyword. Thus, another way you can increase the efficiency of your code is to
use a single InstallPkgs keyword to install multiple packages, as seen below:

\fvset{hllines={, ,}}%
\begin{sphinxVerbatim}[commandchars=\\\{\}]
DistType \PYGZdq{}debian\PYGZdq{}

MirrorURL \PYGZdq{}http://us.archive.ubuntu.com/ubuntu/\PYGZdq{}

OSVersion \PYGZdq{}trusty\PYGZdq{}


Setup

Bootstrap


InstallPkgs python wget

RunCmd /bin/sh \PYGZlt{}\PYGZlt{}EOF

wget https://bootstrap.pypa.io/get\PYGZhy{}pip.py

python get\PYGZhy{}pip.py

ln \PYGZhy{}s /usr/local/bin/pip /usr/bin/pip

pip install \PYGZhy{}\PYGZhy{}upgrade https://storage.googleapis.com/tensorflow/linux/cpu/tensorflow\PYGZhy{}0.9.0\PYGZhy{}cp27\PYGZhy{}none\PYGZhy{}linux\PYGZus{}x86\PYGZus{}64.whl

EOF


Cleanup
\end{sphinxVerbatim}

Using a single InstallPkgs keyword to install both ‘python’ and ‘wget’ allowed to decrease the number of InstallPkgs keywords we had to use in our definition file.
This slimmed down our definition file and helped reduce clutter.



\renewcommand{\indexname}{Index}
\printindex
\end{document}